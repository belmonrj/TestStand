\documentclass[compress,10pt]{beamer} %%%
\usetheme{Warsaw} %%%
%\usecolortheme{cu} %%%

%\setbeamersize{text margin left=10pt,text margin right=10pt}

\usepackage{changepage}

\newcommand{\lenitem}[2][.7\linewidth]{\parbox[t]{#1}{\strut #2\strut}}

%\newcommand\Fontvi{\fontsize{6}{7.2}\selectfont} %%% doesn't hel for subbullets :(
%\setlength{\columnsep}{-20pt} % doesn't work

%%% define my own headline
\setbeamertemplate{headline}
{%
  \leavevmode%
  \begin{beamercolorbox}[wd=\paperwidth,ht=2.5ex,dp=1.125ex]{section in head/foot}%
    \insertsectionnavigationhorizontal{\paperwidth}{\hskip0pt}{}%
  \end{beamercolorbox}%
  \vskip2pt%
}

%%% define my own footline
\setbeamertemplate{footline}
{%
  \leavevmode%
  \hbox{\begin{beamercolorbox}[wd=.5\paperwidth,ht=2.5ex,dp=1.125ex,leftskip=.3cm plus1fill,rightskip=.3cm]{author in head/foot}%
    \usebeamerfont{author in head/foot}\insertshortauthor
  \end{beamercolorbox}%
  \begin{beamercolorbox}[wd=.5\paperwidth,ht=2.5ex,dp=1.125ex,leftskip=.3cm,rightskip=.3cm plus1fil]{title in head/foot}%
    \usebeamerfont{title in head/foot}\insertshorttitle
  \end{beamercolorbox}}%
  \vskip0pt%
}

%%% set another another option
\setbeamertemplate{navigation symbols}{} % this is the same as \beamertemplatenavigationsymbolsempty

%%% add some packages
\usepackage{amsmath,amssymb,amsfonts,amsthm,amsbsy} % math symbols and the like
\usepackage{slashed} % Feynman slashed notatation
\usepackage{tikz}
\usepackage{pdfpages} % include pdfs as a whole page

%%% Cyrillic fonts
\input cyracc.def
%\font\tencyr=wncyr10
\font\tencyr=wncyr9
\def\cyr{\tencyr\cyracc}

%makes footnote with symbol instead of number
\long\def\symbolfootnote[#1]#2{\begingroup\def\thefootnote{\fnsymbol{footnote}}\footnote[#1]{#2}\endgroup}

%\symbolfootnote[sym #]{footnote text}     %makes footnote with symbol instead of number
%\long\def\symbolfootnote[#1]#2{\begingroup
%\def\thefootnote{\fnsymbol{footnote}}\footnotetext[#1]{#2}\endgroup}

\newcommand{\pd}[2]{\frac{\partial #1}{\partial #2}} %% Creates a partial derivative
\newcommand{\totd}[2]{\frac{d #1}{d #2}} %% Creates a total derivative


%%% title and basic information
%\title[sPHENIX HCal meeting, Oct 20, 2015 - Slide \insertframenumber]{First results from the big tile}
%\title[Jan 29, 2016 - Slide \insertframenumber]{Large HCal Tile Testing at Colorado}
%\author[CU-Boulder]{Sebastian Vazquez-Torres \\  Ron Belmont \\ Jamie Nagle \\ \vspace{20pt} University of Colorado, Boulder}
%\date{January 29\textsuperscript{th}, 2016}
\title[\today - Slide \insertframenumber]{High $\eta$ HCal Tile Testing at Colorado}
\author[CU-Boulder]{Sebastian and Ron \\ \vspace{20pt} University of Colorado Boulder}
\date{\today}


%%%%%%%%%%%%%%%%%%%%%%%%%%%%%% remove line above footnotes...
\renewcommand{\footnoterule}{
  \kern -3pt
  \hrule width \textwidth height 0pt
  \kern 3pt
}


%\titlegraphic{\includegraphics[height=1.0cm]{logos/sphenixlogo.pdf}\hfill\includegraphics[trim=0 44 0 30, clip=true, height=1.0cm]{logos/Boulder_FL.pdf}}
%\titlegraphic{\includegraphics[height=1.0cm]{logos/sphenixlogo.pdf}\hfill\includegraphics[trim=0 30 0 30, clip=true, height=1.0cm]{logos/Boulder_FL.pdf}}
\titlegraphic{\includegraphics[height=1.2cm]{logos/sphenixlogo.pdf}\hfill\includegraphics[trim=0 30 0 30, clip=true, height=1.2cm]{logos/Boulder_FL.pdf}}


%%% - now begin document
\begin{document}


%%% - title slide
\begin{frame}
\titlepage
\end{frame}




\begin{frame}{Tile Scan}
\begin{adjustwidth}{-3em}{-3em}
\begin{center}
\includegraphics[width=0.85\linewidth]{./Photos/FarEtaOuterHCalScanBrokenFiber.png}
\end{center}
\end{adjustwidth}
Scan of farthest eta OHCal tile \\
What's going on here???
\end{frame}


\begin{frame}{Broken Fiber}
\begin{adjustwidth}{-3em}{-3em}
\begin{center}
\includegraphics[width=0.85\linewidth]{./Photos/20170419_131755.jpg}
\end{center}
\end{adjustwidth}
Cracks in the fiber near the SiPM
\end{frame}


\begin{frame}{Minor damage to shipping box}
\begin{adjustwidth}{-3em}{-3em}
\begin{center}
\includegraphics[width=0.75\linewidth]{./Photos/IMG_4589.jpg}
\end{center}
\end{adjustwidth}
Minor damage to shipping box but no obvious damage to tile
\end{frame}


\begin{frame}{Condition of tile}
\begin{adjustwidth}{-3em}{-3em}
\begin{center}
\includegraphics[width=0.75\linewidth]{./Photos/IMG_4566.jpg}
\end{center}
\end{adjustwidth}
Minor damage to outer foil on corner (likely prior to shipping)
\end{frame}


\begin{frame}{Condition of tile}
\begin{adjustwidth}{-3em}{-3em}
\begin{center}
\includegraphics[width=0.75\linewidth]{./Photos/IMG_4568.jpg}
\end{center}
\end{adjustwidth}
No damage to inner foil this side
\end{frame}

\begin{frame}{Condition of tile}
\begin{adjustwidth}{-3em}{-3em}
\begin{center}
\includegraphics[width=0.75\linewidth]{./Photos/IMG_4567.jpg}
\end{center}
\end{adjustwidth}
Incomplete inner foil coverage on this side but no visible damage
\end{frame}

\begin{frame}{Condition of tile}
\begin{adjustwidth}{-3em}{-3em}
\begin{center}
\includegraphics[width=0.75\linewidth]{./Photos/IMG_4590.jpg}
\end{center}
\end{adjustwidth}
No visible damage on unwrapped tile
\end{frame}

\begin{frame}{Condition of tile}
\begin{adjustwidth}{-3em}{-3em}
\begin{center}
\includegraphics[width=0.75\linewidth]{./Photos/IMG_4591.jpg}
\end{center}
\end{adjustwidth}
No visible damage on unwrapped tile
\end{frame}

\begin{frame}{Condition of tile}
\begin{adjustwidth}{-3em}{-3em}
\begin{center}
\includegraphics[width=0.75\linewidth]{./Photos/IMG_4573.jpg}
\end{center}
\end{adjustwidth}
Coating rubs off very easily, not clear if this is a problem
\end{frame}


\begin{frame}{Things to consider}
When did the crack happen?
\begin{itemize}
\item The shipping box containing the tiles was packed very carefully with lots of foam surrounding the tiles on all sides
\item The wrapping (outer black foil, inner plastic wrap, inner metal foil) showed no signs of damage near the fiber
\item Visual inspection of unwrapped the tile showed no signs of damage
\item Analysis of the tile mapper data should help determine whether the damage occurred before or after the shipment from BNL to CU
\end{itemize}
\end{frame}





\end{document}

