\documentclass[compress,8pt]{beamer} %%%
\usetheme{Warsaw} %%%
\usecolortheme{cu} %%%

%\setbeamersize{text margin left=10pt,text margin right=10pt}

\usepackage{changepage}

\newcommand{\lenitem}[2][.7\linewidth]{\parbox[t]{#1}{\strut #2\strut}}

%\newcommand\Fontvi{\fontsize{6}{7.2}\selectfont} %%% doesn't hel for subbullets :(
%\setlength{\columnsep}{-20pt} % doesn't work

%%% define my own headline
\setbeamertemplate{headline}
{%
  \leavevmode%
  \begin{beamercolorbox}[wd=\paperwidth,ht=2.5ex,dp=1.125ex]{section in head/foot}%
    \insertsectionnavigationhorizontal{\paperwidth}{\hskip0pt}{}%
  \end{beamercolorbox}%
  \vskip2pt%
}

%%% define my own footline
\setbeamertemplate{footline}
{%
  \leavevmode%
  \hbox{\begin{beamercolorbox}[wd=.5\paperwidth,ht=2.5ex,dp=1.125ex,leftskip=.3cm plus1fill,rightskip=.3cm]{author in head/foot}%
    \usebeamerfont{author in head/foot}\insertshortauthor
  \end{beamercolorbox}%
  \begin{beamercolorbox}[wd=.5\paperwidth,ht=2.5ex,dp=1.125ex,leftskip=.3cm,rightskip=.3cm plus1fil]{title in head/foot}%
    \usebeamerfont{title in head/foot}\insertshorttitle
  \end{beamercolorbox}}%
  \vskip0pt%
}

%%% set another another option
\setbeamertemplate{navigation symbols}{} % this is the same as \beamertemplatenavigationsymbolsempty

%%% add some packages
\usepackage{amsmath,amssymb,amsfonts,amsthm,amsbsy} % math symbols and the like
\usepackage{slashed} % Feynman slashed notatation
\usepackage{tikz}
\usepackage{pdfpages} % include pdfs as a whole page

%%% Cyrillic fonts
\input cyracc.def
%\font\tencyr=wncyr10
\font\tencyr=wncyr9
\def\cyr{\tencyr\cyracc}

%makes footnote with symbol instead of number
\long\def\symbolfootnote[#1]#2{\begingroup\def\thefootnote{\fnsymbol{footnote}}\footnote[#1]{#2}\endgroup}

%\symbolfootnote[sym #]{footnote text}     %makes footnote with symbol instead of number
%\long\def\symbolfootnote[#1]#2{\begingroup
%\def\thefootnote{\fnsymbol{footnote}}\footnotetext[#1]{#2}\endgroup}

\newcommand{\pd}[2]{\frac{\partial #1}{\partial #2}} %% Creates a partial derivative
\newcommand{\totd}[2]{\frac{d #1}{d #2}} %% Creates a total derivative


%%% title and basic information
%\title[sPHENIX HCal meeting, Oct 20, 2015 - Slide \insertframenumber]{First results from the big tile}
\title[Jan 12, 2016 - Slide \insertframenumber]{HCal Tile Testing at Colorado}
\author[CU-Boulder]{Sebastian Vazquez-Torres \\  Ron Belmont \\ Jamie Nagle \\ \vspace{20pt} University of Colorado, Boulder}
\date{January 12\textsuperscript{th}, 2016}


%%%%%%%%%%%%%%%%%%%%%%%%%%%%%% remove line above footnotes...
\renewcommand{\footnoterule}{
  \kern -3pt
  \hrule width \textwidth height 0pt
  \kern 3pt
}


%\titlegraphic{\includegraphics[height=1.0cm]{logos/sphenixlogo.pdf}\hfill\includegraphics[trim=0 44 0 30, clip=true, height=1.0cm]{logos/Boulder_FL.pdf}}
%\titlegraphic{\includegraphics[height=1.0cm]{logos/sphenixlogo.pdf}\hfill\includegraphics[trim=0 30 0 30, clip=true, height=1.0cm]{logos/Boulder_FL.pdf}}
\titlegraphic{\includegraphics[height=1.2cm]{logos/sphenixlogo.pdf}\hfill\includegraphics[trim=0 30 0 30, clip=true, height=1.2cm]{logos/Boulder_FL.pdf}}


%%% - now begin document
\begin{document}


%%% - title slide
\begin{frame}
\titlepage
\end{frame}









\begin{frame}{Large tile LED scan with 405 nm}
\begin{adjustwidth}{-3em}{-3em}
\begin{center}
%\includegraphics[width=\linewidth]{../Figures/Burn/20151026-1704_VMIN_SIPM1_meanHistSub.pdf}
\includegraphics[width=\linewidth]{../Figures/Burn/20160113-1238_VMIN_SIPM1_meanHistSub.pdf}
\end{center}
\end{adjustwidth}
405 nm LED \\
We scan in 0.5 cm increments in both directions \\
We scan 50 rows and 174 columns, which is slightly bigger than the tile itself
\end{frame}



\begin{frame}{Large tile LED scan with 375 nm}
\begin{adjustwidth}{-3em}{-3em}
\begin{center}
\includegraphics[width=\linewidth]{../Figures/Burn/20160111-1335_VMIN_SIPM1_meanHistSub.pdf}
\end{center}
\end{adjustwidth}
375 nm LED \\
We scan in 0.5 cm increments in both directions \\
We scan 50 rows and 174 columns, which is slightly bigger than the tile itself
\end{frame}



\begin{frame}{Large tile LED scan with 361 nm}
\begin{adjustwidth}{-3em}{-3em}
\begin{center}
\includegraphics[width=\linewidth]{../Figures/Burn/20160107-1522_VMIN_SIPM1_meanHistSub.pdf}
\end{center}
\end{adjustwidth}
361 nm LED (courtesy of John H.) \\
We scan in 0.5 cm increments in both directions \\
We scan 50 rows and 174 columns, which is slightly bigger than the tile itself
\end{frame}



\begin{frame}{Comparison of 405 nm and 375 nm, projection in the middle}
\begin{adjustwidth}{-3em}{-3em}
\begin{center}
%\includegraphics[width=0.75\linewidth]{../Figures/Burn/20151026-1704_20160111-1335_projectionY_60.pdf}
\includegraphics[width=0.75\linewidth]{../Figures/Burn/20160113-1238_20160111-1335_projectionY_60.pdf}
\end{center}
\end{adjustwidth}
Arbitrary normalization to facilitate shape comparison \\
Surprising similarity between the 405 nm and the 375 nm \\
\end{frame}



\begin{frame}{Comparison of 405 nm and 361 nm, projection in the middle}
\begin{adjustwidth}{-3em}{-3em}
\begin{center}
%\includegraphics[width=0.75\linewidth]{../Figures/Burn/20151026-1704_20160107-1522_projectionY_60.pdf}
\includegraphics[width=0.75\linewidth]{../Figures/Burn/20160113-1238_20160107-1522_projectionY_60.pdf}
\end{center}
\end{adjustwidth}
Arbitrary normalization to facilitate shape comparison \\
Surprising similarity between the 405 nm and the 361 nm \\
\end{frame}



\begin{frame}{Comparison of 405 nm and 361 nm, projection in the middle}
\begin{adjustwidth}{-3em}{-3em}
\begin{center}
\includegraphics[width=0.75\linewidth]{../Figures/Burn/20160111-1335_20160107-1522_projectionY_60.pdf}
\end{center}
\end{adjustwidth}
Arbitrary normalization to facilitate shape comparison \\
The 375 nm and the 361 nm are almost identical \\
\end{frame}



%\begin{frame}{Brief Summary}
%\begin{itemize}
%\item Although the difference is rather small, blue (405 nm) and UV (375 nm and 361 nm) are not exactly the same
%\item The UV has a less pronounced ``valley'' in between the fiber ``peaks''
%\end{itemize}
%\end{frame}



%\begin{frame}
%Extra material
%\end{frame}

%%% DANGER WE DO NOT UNDERSTAND THE CONSTANT OFFSET BACKGROUND HERE %%%
%\begin{frame}{How different is the LED distribution from the source distribution?}
%\begin{adjustwidth}{-3em}{-3em}
%\begin{center}
%\includegraphics[width=0.5\linewidth]{../SlicesOfPizza/middle_gengaus_20151106-1503_VMIN_SIPM1_meanHistSub.pdf}
%\includegraphics[width=0.5\linewidth]{../SlicesOfPizza/slice_sipmB_fit.pdf}
%\end{center}
%\end{adjustwidth}
%Small rectangular panel \\
%1D view of the short side, LED on left, source on right \\
%Similar structure near fiber (exponential), though source a bit wider \\
%completely different far from fiber, with LED continually decaying while source having large contribution everywhere
%\end{frame}



%\begin{frame}{Cosmics}
%\begin{adjustwidth}{-3em}{-3em}
%\begin{center}
%\includegraphics[width=0.75\linewidth]{../Cosmics/20151113-1617_tempLOWffit.pdf}
%\end{center}
%\end{adjustwidth}
%Cosmic trigger in the center of the tile \\
%$[\Delta E]_{MPV}$ = 43.5$\pm$0.5 pe, $\xi$ = 8.0$\pm$0.3 pe---noticeably larger than scans on large
%tile,
%very similar to measurement of cosmics in small rectangular test tile: \\
%$[\Delta E]_{MPV}$ = 43.9$\pm$0.5 pe, $\xi$ = 9.3$\pm$0.2 pe (see presentation from 10/20/15)
%\end{frame}



\end{document}

