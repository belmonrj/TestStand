\documentclass[compress,8pt]{beamer} %%%
\usetheme{Warsaw} %%%

%\setbeamersize{text margin left=10pt,text margin right=10pt}

\usepackage{changepage}

\newcommand{\lenitem}[2][.7\linewidth]{\parbox[t]{#1}{\strut #2\strut}}

%\newcommand\Fontvi{\fontsize{6}{7.2}\selectfont} %%% doesn't hel for subbullets :(
%\setlength{\columnsep}{-20pt} % doesn't work

%%% define my own headline
\setbeamertemplate{headline}
{%
  \leavevmode%
  \begin{beamercolorbox}[wd=\paperwidth,ht=2.5ex,dp=1.125ex]{section in head/foot}%
    \insertsectionnavigationhorizontal{\paperwidth}{\hskip0pt}{}%
  \end{beamercolorbox}%
  \vskip2pt%
}

%%% define my own footline
\setbeamertemplate{footline}
{%
  \leavevmode%
  \hbox{\begin{beamercolorbox}[wd=.5\paperwidth,ht=2.5ex,dp=1.125ex,leftskip=.3cm plus1fill,rightskip=.3cm]{author in head/foot}%
    \usebeamerfont{author in head/foot}\insertshortauthor
  \end{beamercolorbox}%
  \begin{beamercolorbox}[wd=.5\paperwidth,ht=2.5ex,dp=1.125ex,leftskip=.3cm,rightskip=.3cm plus1fil]{title in head/foot}%
    \usebeamerfont{title in head/foot}\insertshorttitle
  \end{beamercolorbox}}%
  \vskip0pt%
}

%%% set another another option
\setbeamertemplate{navigation symbols}{} % this is the same as \beamertemplatenavigationsymbolsempty

%%% add some packages
\usepackage{amsmath,amssymb,amsfonts,amsthm,amsbsy} % math symbols and the like
\usepackage{slashed} % Feynman slashed notatation
\usepackage{tikz}
\usepackage{pdfpages} % include pdfs as a whole page

%%% Cyrillic fonts
\input cyracc.def
%\font\tencyr=wncyr10
\font\tencyr=wncyr9
\def\cyr{\tencyr\cyracc}

%makes footnote with symbol instead of number
\long\def\symbolfootnote[#1]#2{\begingroup\def\thefootnote{\fnsymbol{footnote}}\footnote[#1]{#2}\endgroup}

%\symbolfootnote[sym #]{footnote text}     %makes footnote with symbol instead of number
%\long\def\symbolfootnote[#1]#2{\begingroup
%\def\thefootnote{\fnsymbol{footnote}}\footnotetext[#1]{#2}\endgroup}

\newcommand{\pd}[2]{\frac{\partial #1}{\partial #2}} %% Creates a partial derivative
\newcommand{\totd}[2]{\frac{d #1}{d #2}} %% Creates a total derivative


%%% title and basic information
%\title[sPHENIX HCal meeting, Oct 20, 2015 - Slide \insertframenumber]{First results from the big tile}
\title[Nov 3, 2015 - Slide \insertframenumber]{First results from the big tile}
\author[CU-Boulder]{Sebastian Vazquez-Torres \\  Ron Belmont \\ Jamie Nagle \\ \vspace{20pt} University of Colorado, Boulder}
\date{November 3\textsuperscript{rd}, 2015}


%%%%%%%%%%%%%%%%%%%%%%%%%%%%%% remove line above footnotes...
\renewcommand{\footnoterule}{
  \kern -3pt
  \hrule width \textwidth height 0pt
  \kern 3pt
}


%\titlegraphic{\includegraphics[height=1.0cm]{logos/sphenixlogo.pdf}\hfill\includegraphics[trim=0 44 0 30, clip=true, height=1.0cm]{logos/Boulder_FL.pdf}}
%\titlegraphic{\includegraphics[height=1.0cm]{logos/sphenixlogo.pdf}\hfill\includegraphics[trim=0 30 0 30, clip=true, height=1.0cm]{logos/Boulder_FL.pdf}}
\titlegraphic{\includegraphics[height=1.2cm]{logos/sphenixlogo.pdf}\hfill\includegraphics[trim=0 30 0 30, clip=true, height=1.2cm]{logos/Boulder_FL.pdf}}


%%% - now begin document
\begin{document}


%%% - title slide
\begin{frame}
\titlepage
\end{frame}





\begin{frame}{Setup}
Some photographs of the big tile setup
\begin{adjustwidth}{-2em}{-2em}
\begin{center}
\includegraphics[width=0.5\linewidth]{../Photos/bigupper.jpg}
\includegraphics[width=0.5\linewidth]{../Photos/biglower.jpg}
\end{center}
\end{adjustwidth}
We have both phototubes under the table so that LED scans can be run without total deconstruction \\
We have the upper tube as close to the tile as possible to minimize the fraction of particles that
trigger both tubes but miss the tile
\end{frame}





\begin{frame}{Cosmics}
\begin{adjustwidth}{-2em}{-2em}
\begin{center}
\includegraphics[width=0.75\linewidth]{../Cosmics/20151029-1340_tempLOWffit.pdf}
\includegraphics[width=0.5\linewidth,angle=90]{../PanelFigures/bigpanel107_box4.pdf}
\end{center}
\end{adjustwidth}
$[\Delta E]_{MPV} =$ 27.8$\pm$0.5 photoelectrons and $\xi =$ 5.5$\pm$0.3 photoelectrons \\
\end{frame}

\begin{frame}{Cosmics}
\begin{adjustwidth}{-2em}{-2em}
\begin{center}
\includegraphics[width=0.75\linewidth]{../Cosmics/20151022-1538_tempLOWffit.pdf}
\includegraphics[width=0.5\linewidth,angle=90]{../PanelFigures/bigpanel107_box1.pdf}
\end{center}
\end{adjustwidth}
$[\Delta E]_{MPV} =$ 26.2$\pm$0.4 photoelectrons and $\xi =$ 5.1$\pm$0.3 photoelectrons \\
\end{frame}

\begin{frame}{Cosmics}
\begin{adjustwidth}{-2em}{-2em}
\begin{center}
\includegraphics[width=0.75\linewidth]{../Cosmics/20151023-1307_tempLOWffit.pdf}
\includegraphics[width=0.5\linewidth,angle=90]{../PanelFigures/bigpanel107_box2.pdf}
\end{center}
\end{adjustwidth}
$[\Delta E]_{MPV} =$ 21.6$\pm$0.5 photoelectrons and $\xi =$ 5.0$\pm$0.3 photoelectrons \\
\end{frame}

\begin{frame}{Cosmics}
\begin{adjustwidth}{-2em}{-2em}
\begin{center}
\includegraphics[width=0.75\linewidth]{../Cosmics/20151030-1532_tempLOWffit.pdf}
\includegraphics[width=0.5\linewidth,angle=90]{../PanelFigures/bigpanel107_box5.pdf}
\end{center}
\end{adjustwidth}
$[\Delta E]_{MPV} =$ 22.4$\pm$0.6 photoelectrons and $\xi =$ 5.2$\pm$0.3 photoelectrons \\
\end{frame}

\begin{frame}{Cosmics}
\begin{adjustwidth}{-2em}{-2em}
\begin{center}
\includegraphics[width=0.75\linewidth]{../Cosmics/20151027-1804_tempLOWffit.pdf}
\includegraphics[width=0.5\linewidth,angle=90]{../PanelFigures/bigpanel107_box3.pdf}
\end{center}
\end{adjustwidth}
$[\Delta E]_{MPV} =$ 19.8$\pm$0.7 photoelectrons and $\xi =$ 5.0$\pm$0.3 photoelectrons \\
\end{frame}





\begin{frame}{LED scan}
\begin{adjustwidth}{-3em}{-3em}
\begin{center}
\includegraphics[width=\linewidth]{../Figures/Burn/20151026-1704_VMIN_SIPM1_meanHistSub.pdf}
\end{center}
\end{adjustwidth}
We scan in 0.5 cm increments in both directions \\
We scan 50 rows and 174 columns, which is slightly bigger than the tile itself
\end{frame}





\begin{frame}{LED scan}
\begin{adjustwidth}{-3em}{-3em}
\begin{center}
\includegraphics[width=0.75\linewidth]{../Figures/Burn/20151026-1704_VMIN_SIPM1_meanHistSub_projectionY_5.pdf}
\includegraphics[width=0.5\linewidth,angle=90]{../PanelFigures/bigpanel107_proj_5.pdf}
\end{center}
\end{adjustwidth}
1D view of the short side, very close to the SiPM
\end{frame}


\begin{frame}{LED scan}
\begin{adjustwidth}{-3em}{-3em}
\begin{center}
\includegraphics[width=0.75\linewidth]{../Figures/Burn/20151026-1704_VMIN_SIPM1_meanHistSub_projectionY_15.pdf}
\includegraphics[width=0.5\linewidth,angle=90]{../PanelFigures/bigpanel107_proj_15.pdf}
\end{center}
\end{adjustwidth}
1D view of the short side, close to the SiPM
\end{frame}


\begin{frame}{LED scan}
\begin{adjustwidth}{-3em}{-3em}
\begin{center}
\includegraphics[width=0.75\linewidth]{../Figures/Burn/20151026-1704_VMIN_SIPM1_meanHistSub_projectionY_30.pdf}
\includegraphics[width=0.5\linewidth,angle=90]{../PanelFigures/bigpanel107_proj_30.pdf}
\end{center}
\end{adjustwidth}
1D view of the short side, towards the middle
\end{frame}


\begin{frame}{LED scan}
\begin{adjustwidth}{-3em}{-3em}
\begin{center}
\includegraphics[width=0.75\linewidth]{../Figures/Burn/20151026-1704_VMIN_SIPM1_meanHistSub_projectionY_45.pdf}
\includegraphics[width=0.5\linewidth,angle=90]{../PanelFigures/bigpanel107_proj_45.pdf}
\end{center}
\end{adjustwidth}
1D view of the short side, towards the middle
\end{frame}


\begin{frame}{LED scan}
\begin{adjustwidth}{-3em}{-3em}
\begin{center}
\includegraphics[width=0.75\linewidth]{../Figures/Burn/20151026-1704_VMIN_SIPM1_meanHistSub_projectionY_60.pdf}
\includegraphics[width=0.5\linewidth,angle=90]{../PanelFigures/bigpanel107_proj_60.pdf}
\end{center}
\end{adjustwidth}
1D view of the short side, towards the middle
\end{frame}


\begin{frame}{LED scan}
\begin{adjustwidth}{-3em}{-3em}
\begin{center}
\includegraphics[width=0.75\linewidth]{../Figures/Burn/20151026-1704_VMIN_SIPM1_meanHistSub_projectionY_87.pdf}
\includegraphics[width=0.5\linewidth,angle=90]{../PanelFigures/bigpanel107_proj_87.pdf}
\end{center}
\end{adjustwidth}
1D view of the short side, middle bin along the long axis
\end{frame}


\begin{frame}{LED scan}
\begin{adjustwidth}{-3em}{-3em}
\begin{center}
\includegraphics[width=0.75\linewidth]{../Figures/Burn/20151026-1704_VMIN_SIPM1_meanHistSub_projectionY_120.pdf}
\includegraphics[width=0.5\linewidth,angle=90]{../PanelFigures/bigpanel107_proj_120.pdf}
\end{center}
\end{adjustwidth}
1D view of the short side, towards the opposite side
\end{frame}


\begin{frame}{LED scan}
\begin{adjustwidth}{-3em}{-3em}
\begin{center}
\includegraphics[width=0.75\linewidth]{../Figures/Burn/20151026-1704_VMIN_SIPM1_meanHistSub_projectionY_150.pdf}
\includegraphics[width=0.5\linewidth,angle=90]{../PanelFigures/bigpanel107_proj_150.pdf}
\end{center}
\end{adjustwidth}
1D view of the short side, towards the opposite side
\end{frame}


\begin{frame}{LED scan}
\begin{adjustwidth}{-3em}{-3em}
\begin{center}
\includegraphics[width=0.75\linewidth]{../Figures/Burn/20151026-1704_VMIN_SIPM1_meanHistSub_projectionY_160.pdf}
\includegraphics[width=0.5\linewidth,angle=90]{../PanelFigures/bigpanel107_proj_160.pdf}
\end{center}
\end{adjustwidth}
1D view of the short side, towards the opposite side
\end{frame}


\begin{frame}{LED scan}
\begin{adjustwidth}{-3em}{-3em}
\begin{center}
\includegraphics[width=0.75\linewidth]{../Figures/Burn/20151026-1704_VMIN_SIPM1_meanHistSub_projectionY_166.pdf}
\includegraphics[width=0.5\linewidth,angle=90]{../PanelFigures/bigpanel107_proj_166.pdf}
\end{center}
\end{adjustwidth}
1D view of the short side, far edge of the panel
\end{frame}


\begin{frame}{Next steps}
\begin{itemize}
\item Suggestions welcome!
\item Do we have enough cosmics on the large tile?
\item Cosmics and LED tests on the small tiles can also be done
\item If we're happy with the large tile results, we can move on to the small tiles
\end{itemize}
\end{frame}





\end{document}

