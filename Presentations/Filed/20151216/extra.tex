\documentclass[compress,8pt]{beamer} %%%
\usetheme{Warsaw} %%%

%\setbeamersize{text margin left=10pt,text margin right=10pt}

\usepackage{changepage}

\newcommand{\lenitem}[2][.7\linewidth]{\parbox[t]{#1}{\strut #2\strut}}

%\newcommand\Fontvi{\fontsize{6}{7.2}\selectfont} %%% doesn't hel for subbullets :(
%\setlength{\columnsep}{-20pt} % doesn't work

%%% define my own headline
\setbeamertemplate{headline}
{%
  \leavevmode%
  \begin{beamercolorbox}[wd=\paperwidth,ht=2.5ex,dp=1.125ex]{section in head/foot}%
    \insertsectionnavigationhorizontal{\paperwidth}{\hskip0pt}{}%
  \end{beamercolorbox}%
  \vskip2pt%
}

%%% define my own footline
\setbeamertemplate{footline}
{%
  \leavevmode%
  \hbox{\begin{beamercolorbox}[wd=.5\paperwidth,ht=2.5ex,dp=1.125ex,leftskip=.3cm plus1fill,rightskip=.3cm]{author in head/foot}%
    \usebeamerfont{author in head/foot}\insertshortauthor
  \end{beamercolorbox}%
  \begin{beamercolorbox}[wd=.5\paperwidth,ht=2.5ex,dp=1.125ex,leftskip=.3cm,rightskip=.3cm plus1fil]{title in head/foot}%
    \usebeamerfont{title in head/foot}\insertshorttitle
  \end{beamercolorbox}}%
  \vskip0pt%
}

%%% set another another option
\setbeamertemplate{navigation symbols}{} % this is the same as \beamertemplatenavigationsymbolsempty

%%% add some packages
\usepackage{amsmath,amssymb,amsfonts,amsthm,amsbsy} % math symbols and the like
\usepackage{slashed} % Feynman slashed notatation
\usepackage{tikz}
\usepackage{pdfpages} % include pdfs as a whole page

%%% Cyrillic fonts
\input cyracc.def
%\font\tencyr=wncyr10
\font\tencyr=wncyr9
\def\cyr{\tencyr\cyracc}

%makes footnote with symbol instead of number
\long\def\symbolfootnote[#1]#2{\begingroup\def\thefootnote{\fnsymbol{footnote}}\footnote[#1]{#2}\endgroup}

%\symbolfootnote[sym #]{footnote text}     %makes footnote with symbol instead of number
%\long\def\symbolfootnote[#1]#2{\begingroup
%\def\thefootnote{\fnsymbol{footnote}}\footnotetext[#1]{#2}\endgroup}

\newcommand{\pd}[2]{\frac{\partial #1}{\partial #2}} %% Creates a partial derivative
\newcommand{\totd}[2]{\frac{d #1}{d #2}} %% Creates a total derivative


%%% title and basic information
\title[sPHENIX HCal meeting, Oct 20, 2015 - Slide \insertframenumber]{Test Stand Status}
\author[CU-Boulder]{Sebastian Vazquez-Torres \\  Ron Belmont \\ Jamie Nagle \\ \vspace{20pt} University of Colorado, Boulder}
\date{October 20\textsuperscript{th}, 2015}


%%%%%%%%%%%%%%%%%%%%%%%%%%%%%% remove line above footnotes...
\renewcommand{\footnoterule}{
  \kern -3pt
  \hrule width \textwidth height 0pt
  \kern 3pt
}


%\titlegraphic{\includegraphics[height=1.0cm]{logos/sphenixlogo.pdf}\hfill\includegraphics[trim=0 44 0 30, clip=true, height=1.0cm]{logos/Boulder_FL.pdf}}
%\titlegraphic{\includegraphics[height=1.0cm]{logos/sphenixlogo.pdf}\hfill\includegraphics[trim=0 30 0 30, clip=true, height=1.0cm]{logos/Boulder_FL.pdf}}
\titlegraphic{\includegraphics[height=1.2cm]{logos/sphenixlogo.pdf}\hfill\includegraphics[trim=0 30 0 30, clip=true, height=1.2cm]{logos/Boulder_FL.pdf}}


%%% - now begin document
\begin{document}


%%% - title slide
\begin{frame}
\titlepage
\end{frame}






\begin{frame}{Cosmics}
Some data about polystyrene (scintillator base material) \\
\begin{tabular}{ll}
\hline
$Z/A$                                                 & 0.53768 mol g$^{-1}$ \\
%$Z$                                                   & 6.0000               \\
%$A$                                                   & 11.1592 g mol$^{-1}$ \\
Density $\rho$                                        & 1.060 g cm$^{-3}$    \\
Nuclear interaction length $\lambda_I$                & 77.1 cm              \\
Radiation length $X_0$                                & 41.31 cm             \\
Mean excitation energy $I$                            & 68.7 eV              \\
Minimum ionization energy $dE/dx|_{min}$              & 2.025 MeV/cm         \\
\hline
\end{tabular}
\end{frame}



\begin{frame}{Cosmics}
What about the Bethe forumla?
\begin{equation*}
-\left<\frac{dE}{dx}\right> = \rho K\frac{Z}{A}\frac{1}{\beta^2}
\left[ \frac{1}{2} \ln \frac{2m_ec^2\beta^2\gamma^2T_{max}}{I^2} - \beta^2 - \frac{\delta(\beta\gamma)}{2}\right]
\end{equation*}
\begin{itemize}
%\item $K = 4\pi N_A r_e^2 m_ec^2 = 0.307075~\textnormal{MeV}~\textnormal{g}^{-1}~\textnormal{cm}^2$
\item $K = 4\pi N_A r_e^2 m_ec^2 = 0.307075~\textnormal{MeV}~\textnormal{mol}^{-1}~\textnormal{cm}^2$
\item $T_{max} = 2m_ec^2\beta^2\gamma^2/[1 + 2\gamma m_e/M + (m_e/M)^2]$
is the highest kinetic energy that can be imparted to a free electron in a single collision
by a particle with mass $M$
\item The $\delta(\beta\gamma)$ can be calculated based on Sternheimer et al, Phys. Rev. B 26, 6067
(1982)---note that in their convention the correction doesn't have the factor 1/2 \\
$X = \log_{10} \beta\gamma$ \\
$X < X_0 \rightarrow \delta(X) = 0$ \\
$X_0 < X < X_1 \rightarrow \delta(X) = 4.6052X + a(X_1-X)^m + C$ \\
$X_1 < X \rightarrow \delta(X) = 4.6052X + C$ \\
where $X_0$, $X_1$, $a$, $m$, and $C$ are material-specific constants
\item Minimum ionizing energy for muon is 318 MeV, using this we obtain $\langle dE/dx\rangle$ = 2.036 MeV/cm,
which is quite close to the 2.025 MeV/cm from the data table
\end{itemize}
\end{frame}



\begin{frame}{Cosmics}
Generally we consider the distribution of energies to be a Landau \\
The Landau distribution is defined by
\begin{equation*}
f(\lambda) = \frac{1}{2\pi i} \int_{c+i\infty}^{c+i\infty} e^{s \ln s + \lambda s} ds
\end{equation*}
due to the long tail, the moments are undefined. \\
\vspace{10pt}
However, as derived in  J. E. Moyal, Phil. Mag. 46 (1955) 263, the Landau distribution can be well approximated by
\begin{equation*}
f(\lambda) = \frac{1}{\sqrt{2\pi}} e^{-\lambda+e^{-\lambda}}
\end{equation*}
which later became known as the Gumbel distribution. \\
The moments of the Gumbel distibution are defined:
%for $\lambda = (x-\mu)/\sigma$, the mode (MPV) is $\mu$ and the mean is $\mu + 0.577216\sigma$ \\
for $\lambda = (x-\mu)/\sigma$, the mode (MPV) is $\mu$ and the mean is $\mu + \gamma_E\sigma$
(where $\gamma_E \approx$ 0.577216 is the Euler-Mascheroni constant)
%\vspace{10pt}
%\tiny{
%NB---Gumbel was a mathematician and Anti-Nazi political writer, exiled from Germany in 1932
%}
\end{frame}



\begin{frame}{Cosmics}
A little about the geometry...
\begin{itemize}
\item The two phototubes are separated by about 3.25", and each has a 1" $\times$ 1" scintillator
\item The maximum angle off-normal is
$\theta_{max} = \tan^{-1}(3.25/1) \approx 0.298 \approx 17^{\circ} $
\item The pathlength $L$ is related to the thickness $T$ by $L = T \sec \theta$,
meaning $L_{max} = T \sec \theta_{max} \approx 1.05T$
\item Panel thickness 0.300" = 0.762 cm
\item For the sake of simplicity, we'll ignore the 5\% possible deviation in pathlength...
%and assume $E_{loss,min} = dE/dx|_{min} \times T =$ 1.56 MeV
\end{itemize}
\end{frame}



\begin{frame}{Cosmics}
\begin{adjustwidth}{-2em}{-2em}
\begin{center}
\includegraphics[width=0.75\linewidth]{../Cosmics/20151009-1743_temp.pdf}
\end{center}
\end{adjustwidth}
\end{frame}



\begin{frame}{Cosmics}
\begin{adjustwidth}{-2em}{-2em}
\begin{center}
\includegraphics[width=0.75\linewidth]{../Cosmics/20151009-1743_templow.pdf}
\end{center}
\end{adjustwidth}
\end{frame}



\begin{frame}{Cosmics}
\begin{adjustwidth}{-2em}{-2em}
\begin{center}
\includegraphics[width=0.75\linewidth]{../Cosmics/20151009-1743_templowffit.pdf}
\end{center}
\end{adjustwidth}
\end{frame}



\begin{frame}{Cosmics and Source}
\begin{adjustwidth}{-2em}{-2em}
\begin{center}
\includegraphics[width=0.5\linewidth]{../Cosmics/20151009-1743_templow.pdf}
\includegraphics[width=0.5\linewidth]{../Source/source_sum.pdf}
\end{center}
\end{adjustwidth}
Raw distributions
\end{frame}



\begin{frame}{Cosmics and Source}
\begin{adjustwidth}{-2em}{-2em}
\begin{center}
\includegraphics[width=0.5\linewidth]{../Cosmics/20151009-1743_templowfit.pdf}
\includegraphics[width=0.5\linewidth]{../Source/source_gaus_sum.pdf}
\end{center}
\end{adjustwidth}
Built-in Landau for cosmics, Built-in Gaussian for source
\end{frame}



\begin{frame}{Cosmics and Source}
\begin{adjustwidth}{-2em}{-2em}
\begin{center}
\includegraphics[width=0.5\linewidth]{../Cosmics/20151009-1743_templowffit.pdf}
\includegraphics[width=0.5\linewidth]{../Source/source_lskewgengaus_sum.pdf}
\end{center}
\end{adjustwidth}
Gumbel for cosmics, modified Gaussian for source
\end{frame}



\begin{frame}{Source}
What does the Strontium source distribution look like?
\begin{adjustwidth}{-2em}{-2em}
\begin{center}
\includegraphics[width=0.5\linewidth]{../Source/strontium_kurie.pdf}
\includegraphics[width=0.5\linewidth]{../Source/strontium_kurie2p2F.pdf}
\end{center}
\end{adjustwidth}
Data from W. E. Meyerhof, Phys. Rev. 74 (1948) 263
\end{frame}






\end{document}



\begin{frame}{Backgrounds}
\begin{adjustwidth}{-2em}{-2em}
\begin{center}
\includegraphics[width=0.75\linewidth]{../Backgrounds/backgrounds_1v2_log.pdf}
\end{center}
\end{adjustwidth}
\end{frame}


\begin{frame}{Backgrounds}
\begin{adjustwidth}{-2em}{-2em}
\begin{center}
\includegraphics[width=0.75\linewidth]{../Backgrounds/backgrounds_SvA_log.pdf}
\end{center}
\end{adjustwidth}
\end{frame}


\begin{frame}{Cosmics}
\begin{adjustwidth}{-2em}{-2em}
\begin{center}
\includegraphics[width=0.75\linewidth]{../Cosmics/cosmics_1v2_log.pdf}
\end{center}
\end{adjustwidth}
\end{frame}


\begin{frame}{Cosmics}
\begin{adjustwidth}{-2em}{-2em}
\begin{center}
\includegraphics[width=0.75\linewidth]{../Cosmics/cosmics_SvA_log.pdf}
\end{center}
\end{adjustwidth}
\end{frame}






%%%
%%% come back here
%%%



\begin{frame}{Cosmics}
\begin{adjustwidth}{-2em}{-2em}
\begin{center}
\includegraphics<1>[width=0.75\linewidth]{../Cosmics/bhfuglydatabothnotlogfit.pdf}
\includegraphics<2>[width=0.75\linewidth]{../Cosmics/bhfuglydatabothlogfit.pdf}
\end{center}
\end{adjustwidth}
\end{frame}


\begin{frame}{Cosmics}
\begin{adjustwidth}{-2em}{-2em}
\begin{center}
\includegraphics<1>[width=0.75\linewidth]{../Cosmics/pfuglydatabothnotlogfit.pdf}
\includegraphics<2>[width=0.75\linewidth]{../Cosmics/pfuglydatabothlogfit.pdf}
\end{center}
\end{adjustwidth}
\end{frame}




%%%

\begin{frame}{Backgrounds}
\begin{adjustwidth}{-2em}{-2em}
\begin{center}
\includegraphics[width=0.75\linewidth]{../Backgrounds/backgrounds_1v2_log.pdf}
\end{center}
\end{adjustwidth}
\end{frame}


\begin{frame}{Cosmics}
\begin{adjustwidth}{-2em}{-2em}
\begin{center}
\includegraphics[width=0.75\linewidth]{../Cosmics/cosmics_1v2_log.pdf}
\end{center}
\end{adjustwidth}
\end{frame}


\begin{frame}{Cosmics}
\begin{adjustwidth}{-2em}{-2em}
\begin{center}
\includegraphics[width=0.75\linewidth]{../Cosmics/cosmics_1v2_cut2_log.pdf}
\end{center}
\end{adjustwidth}
\end{frame}


\begin{frame}{Cosmics}
\begin{adjustwidth}{-2em}{-2em}
\begin{center}
\includegraphics[width=0.75\linewidth]{../Cosmics/cosmics_1v2_cut1_log.pdf}
\end{center}
\end{adjustwidth}
\end{frame}


\begin{frame}{Backgrounds}
\begin{adjustwidth}{-2em}{-2em}
\begin{center}
\includegraphics[width=0.75\linewidth]{../Backgrounds/backgrounds_SvA_log.pdf}
\end{center}
\end{adjustwidth}
\end{frame}


\begin{frame}{Cosmics}
\begin{adjustwidth}{-2em}{-2em}
\begin{center}
\includegraphics[width=0.75\linewidth]{../Cosmics/cosmics_SvA_log.pdf}
\end{center}
\end{adjustwidth}
\end{frame}


\begin{frame}{Cosmics}
\begin{adjustwidth}{-2em}{-2em}
\begin{center}
\includegraphics[width=0.75\linewidth]{../Cosmics/cosmics_SvA_cut2_log.pdf}
\end{center}
\end{adjustwidth}
\end{frame}


\begin{frame}{Cosmics}
\begin{adjustwidth}{-2em}{-2em}
\begin{center}
\includegraphics[width=0.75\linewidth]{../Cosmics/cosmics_SvA_cut1_log.pdf}
\end{center}
\end{adjustwidth}
\end{frame}


%\begin{frame}{Source}
%\begin{adjustwidth}{-2em}{-2em}
%\begin{center}
%\includegraphics[width=0.75\linewidth]{../Source/source_1v2_log.pdf}
%\end{center}
%\end{adjustwidth}
%\end{frame}


%\begin{frame}{Source}
%\begin{adjustwidth}{-2em}{-2em}
%\begin{center}
%\includegraphics[width=0.75\linewidth]{../Source/source_SvA_log.pdf}
%\end{center}
%\end{adjustwidth}
%\end{frame}





\begin{frame}{Cosmics}
Some data about polystyrene (scintillator base material) \\
\vspace{10pt}
$Z/A$ = 0.53768 \\
$A$ = 11.1592 \\
Density: 1.060 g cm$^{-3}$ \\
Mean excitation energy: 68.7 eV \\
\begin{tabular}{|l|l|l|l|}
\hline
         & Nucl. int. length $\lambda_I$ & Rad. length $X_0$ & Min. ionization $dE/dx|_{\textnormal{min}}$ \\
\hline
Specific & 81.7 g cm$^{-2}$              & 43.79 g cm$^{-2}$ & 1.936 MeV g$^{-1}$ cm$^2$                   \\
\hline
Standard & 77.1 cm                       & 41.31 cm          & 2.052 MeV/cm                                \\
\hline
\end{tabular}
\vspace{10pt}
Panel thickness 0.300" = 0.762 cm \\
$E_{loss,min} \gtrsim$ 1.56 MeV
\end{frame}


\begin{frame}{Cosmics}
\begin{adjustwidth}{-2em}{-2em}
\begin{center}
\includegraphics[width=0.75\linewidth]{../Cosmics/20151009-1743_tempfit.pdf}
\end{center}
\end{adjustwidth}
\end{frame}



\begin{frame}{Cosmics and Source}
\begin{adjustwidth}{-2em}{-2em}
\begin{center}
\includegraphics[width=0.5\linewidth]{../Cosmics/20151009-1743_templowffit.pdf}
\includegraphics<1>[width=0.5\linewidth]{../Source/source_lgengaus_sum.pdf}
\includegraphics<2>[width=0.5\linewidth]{../Source/source_lskewgengaus_sum.pdf}
\end{center}
\end{adjustwidth}
\end{frame}



\begin{frame}{Cosmics and Source}
\begin{adjustwidth}{-2em}{-2em}
\begin{center}
\includegraphics<1>[width=0.5\linewidth]{../Cosmics/20151009-1743_templow.pdf}
\includegraphics<2>[width=0.5\linewidth]{../Cosmics/20151009-1743_templowfit.pdf}
\includegraphics<3>[width=0.5\linewidth]{../Cosmics/20151009-1743_templowffit.pdf}
\includegraphics<1>[width=0.5\linewidth]{../Source/source_sum.pdf}
\includegraphics<2>[width=0.5\linewidth]{../Source/source_gaus_sum.pdf}
\includegraphics<3>[width=0.5\linewidth]{../Source/source_lskewgengaus_sum.pdf}
\end{center}
\end{adjustwidth}
\end{frame}



