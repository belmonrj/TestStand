\documentclass[compress,8pt]{beamer} %%%
\usetheme{Warsaw} %%%

%\setbeamersize{text margin left=10pt,text margin right=10pt}

\usepackage{changepage}

\newcommand{\lenitem}[2][.7\linewidth]{\parbox[t]{#1}{\strut #2\strut}}

%\newcommand\Fontvi{\fontsize{6}{7.2}\selectfont} %%% doesn't hel for subbullets :(
%\setlength{\columnsep}{-20pt} % doesn't work

%%% define my own headline
\setbeamertemplate{headline}
{%
  \leavevmode%
  \begin{beamercolorbox}[wd=\paperwidth,ht=2.5ex,dp=1.125ex]{section in head/foot}%
    \insertsectionnavigationhorizontal{\paperwidth}{\hskip0pt}{}%
  \end{beamercolorbox}%
  \vskip2pt%
}

%%% define my own footline
\setbeamertemplate{footline}
{%
  \leavevmode%
  \hbox{\begin{beamercolorbox}[wd=.5\paperwidth,ht=2.5ex,dp=1.125ex,leftskip=.3cm plus1fill,rightskip=.3cm]{author in head/foot}%
    \usebeamerfont{author in head/foot}\insertshortauthor
  \end{beamercolorbox}%
  \begin{beamercolorbox}[wd=.5\paperwidth,ht=2.5ex,dp=1.125ex,leftskip=.3cm,rightskip=.3cm plus1fil]{title in head/foot}%
    \usebeamerfont{title in head/foot}\insertshorttitle
  \end{beamercolorbox}}%
  \vskip0pt%
}

%%% set another another option
\setbeamertemplate{navigation symbols}{} % this is the same as \beamertemplatenavigationsymbolsempty

%%% add some packages
\usepackage{amsmath,amssymb,amsfonts,amsthm,amsbsy} % math symbols and the like
\usepackage{slashed} % Feynman slashed notatation
\usepackage{tikz}
\usepackage{pdfpages} % include pdfs as a whole page

%%% Cyrillic fonts
\input cyracc.def
%\font\tencyr=wncyr10
\font\tencyr=wncyr9
\def\cyr{\tencyr\cyracc}

%makes footnote with symbol instead of number
\long\def\symbolfootnote[#1]#2{\begingroup\def\thefootnote{\fnsymbol{footnote}}\footnote[#1]{#2}\endgroup}

%\symbolfootnote[sym #]{footnote text}     %makes footnote with symbol instead of number
%\long\def\symbolfootnote[#1]#2{\begingroup
%\def\thefootnote{\fnsymbol{footnote}}\footnotetext[#1]{#2}\endgroup}

\newcommand{\pd}[2]{\frac{\partial #1}{\partial #2}} %% Creates a partial derivative
\newcommand{\totd}[2]{\frac{d #1}{d #2}} %% Creates a total derivative


%%% title and basic information
\title[sPHENIX HCal meeting, Sept 22, 2015 - Slide \insertframenumber]{Test Stand Status}
\author[CU-Boulder]{Sebastian Vazquez-Torres \\ Sebastian Seeds \\ Ron Belmont \\ Jamie Nagle \\ \vspace{20pt} University of Colorado, Boulder}
\date{September 22\textsuperscript{nd}, 2015}


%%%%%%%%%%%%%%%%%%%%%%%%%%%%%% remove line above footnotes...
\renewcommand{\footnoterule}{
  \kern -3pt
  \hrule width \textwidth height 0pt
  \kern 3pt
}


%\titlegraphic{\includegraphics[height=1.0cm]{logos/sphenixlogo.pdf}\hfill\includegraphics[trim=0 44 0 30, clip=true, height=1.0cm]{logos/Boulder_FL.pdf}}
%\titlegraphic{\includegraphics[height=1.0cm]{logos/sphenixlogo.pdf}\hfill\includegraphics[trim=0 30 0 30, clip=true, height=1.0cm]{logos/Boulder_FL.pdf}}
\titlegraphic{\includegraphics[height=1.2cm]{logos/sphenixlogo.pdf}\hfill\includegraphics[trim=0 30 0 30, clip=true, height=1.2cm]{logos/Boulder_FL.pdf}}


%%% - now begin document
\begin{document}


%%% - title slide
\begin{frame}
\titlepage
\end{frame}

%Figures/SCANCOMP_A1_Source_3Comp_91114-SiPM1_proj_0.png





%




\begin{frame}
Part 1---high voltage supplies and preparing for cosmics
\end{frame}



\begin{frame}{CAEN SY127}
Our primary HV supply is an early '90s CAEN SY127 \\
We have 3 modules on the back, a negative bias A331 (working), a negative bias A333 (not working), and a positive bias A333 (seems to be working...) \\
\vspace{10pt}
\begin{columns}
\begin{column}{0.55\linewidth}
\includegraphics[width=\linewidth]{../Photos/IMG_2638.jpg}
\end{column}
\begin{column}{0.45\linewidth}
\includegraphics[width=0.9\linewidth,angle=270]{../Photos/IMG_2630.jpg}\\
\end{column}
\end{columns}
\vspace{10pt}
Unfortunately, we need a negative bias supply with high current (2--3 mA), and only the A333 has the right specs...
\end{frame}



\begin{frame}{Bertan 380N}
We also have an old Beran 380N laying around... \\
\vspace{10pt}
\begin{columns}
\begin{column}{0.5\linewidth}
\includegraphics[width=\linewidth,angle=270]{../Photos/IMG_2627.jpg}
\end{column}
\begin{column}{0.5\linewidth}
\includegraphics[width=\linewidth,angle=270]{../Photos/IMG_2633.jpg}\\
\end{column}
\end{columns}
\vspace{10pt}
We're not sure if it supplies enough current, but it has the wrong HV jacks \\
Question to the audience: do all Bertans have these jacks, or does it depend on the model/voltage?
\end{frame}





\begin{frame}{CAEN N1470}
We have borrowed a CAEN N1470 from the particle physics group
\begin{center}
\includegraphics[width=0.5\linewidth,angle=270]{../Photos/IMG_2636.jpg}\\
\end{center}
The unit is very nice and has the specs we need, but we can't keep it forever \\
We've requested a quote from CAEN for a new one
\end{frame}





\begin{frame}
Part 2---looking at the distributions from the source
\end{frame}

\begin{frame}{Establishing consistency - Source scans}
Distribution of Foreground and Background
\begin{center}
\includegraphics<1>[width=0.75\linewidth]{../Figures/Distribution/20150915_FGBG_pe.pdf}
\includegraphics<2>[width=0.75\linewidth]{../Figures/Distribution/20150921_FGBG_pe.pdf}
\includegraphics<3>[width=0.75\linewidth]{../Figures/Distribution/20150921_FGBG_pe_sub.pdf}
\includegraphics<4>[width=0.75\linewidth]{../Figures/Distribution/20150921_FGBG_log_pe_sub.pdf}
\end{center}
\only<1>{Event count: 2,000 \\}
\only<2->{Event count: 20,000 \\}
Foreground rate: 1038 Hz \\
Background rate: 27 Hz \\
Background percentage: 2.6\%
\end{frame}


%\end{document}



\begin{frame}
Part 3---establishing consistency in source scans
\end{frame}



\begin{frame}{Establishing consistency - Source scans}
\begin{adjustwidth}{-2em}{-2em}
\includegraphics[width=0.5\linewidth]{../Figures/SCANCOMP_A1_Source_3Comp_91114-SiPM1_proj_0.pdf}
\includegraphics[width=0.5\linewidth]{../Figures/SCANCOMP_A1_Source_3Comp_91114-SiPM2_proj_0.pdf}\\
\end{adjustwidth}
Projection 0, -2.0 cm from fiber
\hfill \includegraphics[width=0.15\linewidth,angle=-90,right]{../PanelFigures/panel2_scanatminus20cm.pdf}
\end{frame}

\begin{frame}{Establishing consistency - Source scans}
\begin{adjustwidth}{-2em}{-2em}
\includegraphics[width=0.5\linewidth]{../Figures/SCANCOMP_A1_Source_3Comp_91114-SiPM1_proj_1.pdf}
\includegraphics[width=0.5\linewidth]{../Figures/SCANCOMP_A1_Source_3Comp_91114-SiPM2_proj_1.pdf}
\end{adjustwidth}
Projection 1, -1.5 cm from fiber
\hfill \includegraphics[width=0.15\linewidth,angle=-90,right]{../PanelFigures/panel2_scanatminus15cm.pdf}
\end{frame}

\begin{frame}{Establishing consistency - Source scans}
\begin{adjustwidth}{-2em}{-2em}
\includegraphics[width=0.5\linewidth]{../Figures/SCANCOMP_A1_Source_3Comp_91114-SiPM1_proj_2.pdf}
\includegraphics[width=0.5\linewidth]{../Figures/SCANCOMP_A1_Source_3Comp_91114-SiPM2_proj_2.pdf}
\end{adjustwidth}
Projection 2, -1.0 cm from fiber
\hfill \includegraphics[width=0.15\linewidth,angle=-90,right]{../PanelFigures/panel2_scanatminus10cm.pdf}
\end{frame}

\begin{frame}{Establishing consistency - Source scans}
\begin{adjustwidth}{-2em}{-2em}
\includegraphics[width=0.5\linewidth]{../Figures/SCANCOMP_A1_Source_3Comp_91114-SiPM1_proj_3.pdf}
\includegraphics[width=0.5\linewidth]{../Figures/SCANCOMP_A1_Source_3Comp_91114-SiPM2_proj_3.pdf}
\end{adjustwidth}
Projection 3, -0.5 cm from fiber
\hfill \includegraphics[width=0.15\linewidth,angle=-90,right]{../PanelFigures/panel2_scanatminus05cm.pdf}
\end{frame}

\begin{frame}{Establishing consistency - Source scans}
\begin{adjustwidth}{-2em}{-2em}
\includegraphics[width=0.5\linewidth]{../Figures/SCANCOMP_A1_Source_3Comp_91114-SiPM1_proj_4.pdf}
\includegraphics[width=0.5\linewidth]{../Figures/SCANCOMP_A1_Source_3Comp_91114-SiPM2_proj_4.pdf}
\end{adjustwidth}
Projection 4, 0.0 cm from fiber
\hfill \includegraphics[width=0.15\linewidth,angle=-90,right]{../PanelFigures/panel2_scanatminus00cm.pdf}
\end{frame}

\begin{frame}{Establishing consistency - Source scans}
\begin{adjustwidth}{-2em}{-2em}
\includegraphics[width=0.5\linewidth]{../Figures/SCANCOMP_A1_Source_3Comp_91114-SiPM1_proj_5.pdf}
\includegraphics[width=0.5\linewidth]{../Figures/SCANCOMP_A1_Source_3Comp_91114-SiPM2_proj_5.pdf}
\end{adjustwidth}
Projection 5, 0.5 cm from fiber
\hfill \includegraphics[width=0.15\linewidth,angle=-90,right]{../PanelFigures/panel2_scanatplus05cm.pdf}
\end{frame}

\begin{frame}{Establishing consistency - Source scans}
\begin{adjustwidth}{-2em}{-2em}
\includegraphics[width=0.5\linewidth]{../Figures/SCANCOMP_A1_Source_3Comp_91114-SiPM1_proj_6.pdf}
\includegraphics[width=0.5\linewidth]{../Figures/SCANCOMP_A1_Source_3Comp_91114-SiPM2_proj_6.pdf}
\end{adjustwidth}
Projection 6, 1.0 cm from fiber
\hfill \includegraphics[width=0.15\linewidth,angle=-90,right]{../PanelFigures/panel2_scanatplus10cm.pdf}
\end{frame}

\begin{frame}{Establishing consistency - Source scans}
\begin{adjustwidth}{-2em}{-2em}
\includegraphics[width=0.5\linewidth]{../Figures/SCANCOMP_A1_Source_3Comp_91114-SiPM1_proj_7.pdf}
\includegraphics[width=0.5\linewidth]{../Figures/SCANCOMP_A1_Source_3Comp_91114-SiPM2_proj_7.pdf}
\end{adjustwidth}
Projection 7, 1.5 cm from fiber
\hfill \includegraphics[width=0.15\linewidth,angle=-90,right]{../PanelFigures/panel2_scanatplus15cm.pdf}
\end{frame}

\begin{frame}{Establishing consistency - Source scans}
\begin{adjustwidth}{-2em}{-2em}
\includegraphics[width=0.5\linewidth]{../Figures/SCANCOMP_A1_Source_3Comp_91114-SiPM1_proj_8.pdf}
\includegraphics[width=0.5\linewidth]{../Figures/SCANCOMP_A1_Source_3Comp_91114-SiPM2_proj_8.pdf}
\end{adjustwidth}
Projection 8, 2.0 cm from fiber
\hfill \includegraphics[width=0.15\linewidth,angle=-90,right]{../PanelFigures/panel2_scanatplus20cm.pdf}
\end{frame}



%%%
%%%
%%%

\begin{frame}
Part 4---light yield modeling
\end{frame}


\begin{frame}{2-component light yield modeling}
Looking at the light yield from each SiPM separately, we can do a simple 2-component model (core+cladding)
\begin{itemize}
\item Let $Y_1$ denote the total light yield in photoelectrons in SiPM1, and similarly for SiPM2 ($Y_2$)
\item Let $N_{\textnormal{core}}$ denote the light in the core,
and similarly for $N_{\textnormal{clad}}$, resulting in \\
$Y_1 = N_{\textnormal{core}} + N_{\textnormal{clad}}$
\item Let $\lambda_{\textnormal{core}}$ denote the attenuation length in the core (3.5 m for Y11 non S-type),
and similarly for $\lambda_{\textnormal{clad}}$ (estimated to be 5 cm)
\item We can model the total light yield in SiPM1 as
$Y_1 = N_{\textnormal{core}} e^{-x/\lambda_{\textnormal{core}}} + N_{\textnormal{clad}}e^{-x/\lambda_{\textnormal{clad}}}$
\item Similarly, and taking into account the local coordinate system where $L$ is the length of the panel,
we can estimate the total light yield in SiPM2 as \\
$Y_2 = N_{\textnormal{core}}e^{(x-L)/\lambda_{\textnormal{core}}} + N_{\textnormal{clad}}e^{(x-L)/\lambda_{\textnormal{clad}}}$
\end{itemize}
\end{frame}



\begin{frame}{2-component light yield modeling}
Looking at the light yield from each SiPM separately, we can do a simple 2-component model (core+cladding)
\begin{itemize}
\item We define the core fraction with the relation $N_{\textnormal{core}} = f_{\textnormal{core}}Y_1$
\item This gives by definition $N_{\textnormal{clad}} = (1-f_{\textnormal{core}})Y_1$
\item We define the asymmetry as $A = (Y_2-Y_1)/(Y_2+Y_1)$ which allows us to cancel the $N$s
and rewrite the light yields as \\
$Y_1 = f_{\textnormal{core}}e^{-x/\lambda_{\textnormal{core}}} + (1-f_{\textnormal{core}})e^{-x/\lambda_{\textnormal{clad}}}$ \\
$Y_2 = f_{\textnormal{core}}e^{(x-L)/\lambda_{\textnormal{core}}} + (1-f_{\textnormal{core}})e^{(x-L)/\lambda_{\textnormal{clad}}}$
\end{itemize}
\end{frame}







%%%
%%%
%%%

\begin{frame}{2-component lightyield}
\begin{adjustwidth}{-2em}{-2em}
\includegraphics[width=0.5\linewidth]{../Figures/FITATOGETHER_20150911-1150_A1_Source_p0.pdf}
\includegraphics[width=0.5\linewidth]{../Figures/FITASYMMETRY_20150911-1150_A1_Source_p0.pdf}
\end{adjustwidth}
Projection 0, -2.0 cm from fiber
\hfill \includegraphics[width=0.15\linewidth,angle=-90,right]{../PanelFigures/panel2_scanatminus20cm.pdf}
\end{frame}

\begin{frame}{2-component lightyield}
\begin{adjustwidth}{-2em}{-2em}
\includegraphics[width=0.5\linewidth]{../Figures/FITATOGETHER_20150911-1150_A1_Source_p1.pdf}
\includegraphics[width=0.5\linewidth]{../Figures/FITASYMMETRY_20150911-1150_A1_Source_p1.pdf}
\end{adjustwidth}
Projection 1, -1.5 cm from fiber
\hfill \includegraphics[width=0.15\linewidth,angle=-90,right]{../PanelFigures/panel2_scanatminus15cm.pdf}
\end{frame}

\begin{frame}{2-component lightyield}
\begin{adjustwidth}{-2em}{-2em}
\includegraphics[width=0.5\linewidth]{../Figures/FITATOGETHER_20150911-1150_A1_Source_p2.pdf}
\includegraphics[width=0.5\linewidth]{../Figures/FITASYMMETRY_20150911-1150_A1_Source_p2.pdf}
\end{adjustwidth}
Projection 2, -1.0 cm from fiber
\hfill \includegraphics[width=0.15\linewidth,angle=-90,right]{../PanelFigures/panel2_scanatminus10cm.pdf}
\end{frame}

\begin{frame}{2-component lightyield}
\begin{adjustwidth}{-2em}{-2em}
\includegraphics[width=0.5\linewidth]{../Figures/FITATOGETHER_20150911-1150_A1_Source_p3.pdf}
\includegraphics[width=0.5\linewidth]{../Figures/FITASYMMETRY_20150911-1150_A1_Source_p3.pdf}
\end{adjustwidth}
Projection 3, -0.5 cm from fiber
\hfill \includegraphics[width=0.15\linewidth,angle=-90,right]{../PanelFigures/panel2_scanatminus05cm.pdf}
\end{frame}

\begin{frame}{2-component lightyield}
\begin{adjustwidth}{-2em}{-2em}
\includegraphics[width=0.5\linewidth]{../Figures/FITATOGETHER_20150911-1150_A1_Source_p4.pdf}
\includegraphics[width=0.5\linewidth]{../Figures/FITASYMMETRY_20150911-1150_A1_Source_p4.pdf}
\end{adjustwidth}
Projection 4, 0.0 cm from fiber
\hfill \includegraphics[width=0.15\linewidth,angle=-90,right]{../PanelFigures/panel2_scanatminus00cm.pdf}
\end{frame}

\begin{frame}{2-component lightyield}
\begin{adjustwidth}{-2em}{-2em}
\includegraphics[width=0.5\linewidth]{../Figures/FITATOGETHER_20150911-1150_A1_Source_p5.pdf}
\includegraphics[width=0.5\linewidth]{../Figures/FITASYMMETRY_20150911-1150_A1_Source_p5.pdf}
\end{adjustwidth}
Projection 5, 0.5 cm from fiber
\hfill \includegraphics[width=0.15\linewidth,angle=-90,right]{../PanelFigures/panel2_scanatplus05cm.pdf}
\end{frame}

\begin{frame}{2-component lightyield}
\begin{adjustwidth}{-2em}{-2em}
\includegraphics[width=0.5\linewidth]{../Figures/FITATOGETHER_20150911-1150_A1_Source_p6.pdf}
\includegraphics[width=0.5\linewidth]{../Figures/FITASYMMETRY_20150911-1150_A1_Source_p6.pdf}
\end{adjustwidth}
Projection 6, 1.0 cm from fiber
\hfill \includegraphics[width=0.15\linewidth,angle=-90,right]{../PanelFigures/panel2_scanatplus10cm.pdf}
\end{frame}

\begin{frame}{2-component lightyield}
\begin{adjustwidth}{-2em}{-2em}
\includegraphics[width=0.5\linewidth]{../Figures/FITATOGETHER_20150911-1150_A1_Source_p7.pdf}
\includegraphics[width=0.5\linewidth]{../Figures/FITASYMMETRY_20150911-1150_A1_Source_p7.pdf}
\end{adjustwidth}
Projection 7, 1.5 cm from fiber
\hfill \includegraphics[width=0.15\linewidth,angle=-90,right]{../PanelFigures/panel2_scanatplus15cm.pdf}
\end{frame}

\begin{frame}{2-component lightyield}
\begin{adjustwidth}{-2em}{-2em}
\includegraphics[width=0.5\linewidth]{../Figures/FITATOGETHER_20150911-1150_A1_Source_p8.pdf}
\includegraphics[width=0.5\linewidth]{../Figures/FITASYMMETRY_20150911-1150_A1_Source_p8.pdf}
\end{adjustwidth}
Projection 8, 2.0 cm from fiber
\hfill \includegraphics[width=0.15\linewidth,angle=-90,right]{../PanelFigures/panel2_scanatplus20cm.pdf}
\end{frame}




\begin{frame}{2-component lightyield}
Comparison of projections
\begin{center}
\includegraphics[width=0.75\linewidth]{../Figures/20150911-1150_A1_Source_AsymmetryComparison.pdf}
\end{center}
\end{frame}


%%%
%%%
%%%

%%% COME BACK HERE

%%%
%%%
%%%

\begin{frame}
Part 5---Parts 3 and 4 but with LED scans
\end{frame}


%Figures/SCANCOMP_20150911-A1_LED_Comp_SiPM1_proj_0.png
\begin{frame}{Establishing consistency - LED scans}
\begin{adjustwidth}{-2em}{-2em}
\includegraphics[width=0.5\linewidth]{../Figures/SCANCOMP_20150911-A1_LED_Comp_SiPM1_proj_0.pdf}
\includegraphics[width=0.5\linewidth]{../Figures/SCANCOMP_20150911-A1_LED_Comp_SiPM2_proj_0.pdf}
\end{adjustwidth}
Projection 0, -1.5 cm from fiber
\hfill \includegraphics[width=0.15\linewidth,angle=-90,right]{../PanelFigures/panel2_scanatminus15cm.pdf}
\end{frame}

\begin{frame}{Establishing consistency - LED scans}
\begin{adjustwidth}{-2em}{-2em}
\includegraphics[width=0.5\linewidth]{../Figures/SCANCOMP_20150911-A1_LED_Comp_SiPM1_proj_1.pdf}
\includegraphics[width=0.5\linewidth]{../Figures/SCANCOMP_20150911-A1_LED_Comp_SiPM2_proj_1.pdf}
\end{adjustwidth}
Projection 1, -1.0 cm from fiber
\hfill \includegraphics[width=0.15\linewidth,angle=-90,right]{../PanelFigures/panel2_scanatminus10cm.pdf}
\end{frame}

\begin{frame}{Establishing consistency - LED scans}
\begin{adjustwidth}{-2em}{-2em}
\includegraphics[width=0.5\linewidth]{../Figures/SCANCOMP_20150911-A1_LED_Comp_SiPM1_proj_2.pdf}
\includegraphics[width=0.5\linewidth]{../Figures/SCANCOMP_20150911-A1_LED_Comp_SiPM2_proj_2.pdf}
\end{adjustwidth}
Projection 2, -0.5 cm from fiber
\hfill \includegraphics[width=0.15\linewidth,angle=-90,right]{../PanelFigures/panel2_scanatminus05cm.pdf}
\end{frame}

\begin{frame}{Establishing consistency - LED scans}
\begin{adjustwidth}{-2em}{-2em}
\includegraphics[width=0.5\linewidth]{../Figures/SCANCOMP_20150911-A1_LED_Comp_SiPM1_proj_3.pdf}
\includegraphics[width=0.5\linewidth]{../Figures/SCANCOMP_20150911-A1_LED_Comp_SiPM2_proj_3.pdf}
\end{adjustwidth}
\only<1>{Projection 3, 0.0 cm from fiber}
\only<2>{LED scans with coated panel are \textbf{\textit{very}} sensitive to alignment}
\hfill \includegraphics[width=0.15\linewidth,angle=-90,right]{../PanelFigures/panel2_scanatminus00cm.pdf} \\
\end{frame}

%\begin{frame}{Establishing consistency - LED scans}
%\begin{adjustwidth}{-2em}{-2em}
%\includegraphics[width=0.5\linewidth]{../Figures/SCANCOMP_20150911-A1_LED_Comp_SiPM1_proj_3.pdf}
%\includegraphics[width=0.5\linewidth]{../Figures/SCANCOMP_20150911-A1_LED_Comp_SiPM2_proj_3.pdf}
%\end{adjustwidth}
%LED scans with coated panel are very sensitive to alignment
%\hfill \includegraphics[width=0.15\linewidth,angle=-90,right]{../PanelFigures/panel2_scanatminus00cm.pdf} \\
%\end{frame}

%\begin{frame}{Establishing consistency - LED scans}
%\begin{adjustwidth}{-2em}{-2em}
%\includegraphics[width=0.5\linewidth]{../Figures/SCANCOMP_20150911-A1_LED_Comp_SiPM1_proj_3.pdf}
%\includegraphics[width=0.5\linewidth]{../Figures/SCANCOMP_20150911-A1_LED_Comp_SiPM2_proj_3.pdf}
%\end{adjustwidth}
%\vspace{0.3cm}
%\begin{adjustwidth}{0.45cm}{0.45cm}
%\begin{columns}
%\begin{column}{0.5\linewidth}
%Projection 3, 0.0 cm from fiber \\
%LED scans with coated panel are extremely sensitive to alignment \\
%\end{column}
%\begin{column}{0.5\linewidth}
%\hfill \includegraphics[width=0.325\linewidth,angle=-90,right]{../PanelFigures/panel2_scanatminus00cm.pdf}
%\end{column}
%\end{columns}
%\end{adjustwidth}
%\end{frame}


%\begin{frame}{Establishing consistency - LED scans}
%\begin{adjustwidth}{-2em}{-2em}
%\includegraphics[width=0.5\linewidth]{../Figures/SCANCOMP_20150911-A1_LED_Comp_SiPM1_proj_3.pdf}
%\includegraphics[width=0.5\linewidth]{../Figures/SCANCOMP_20150911-A1_LED_Comp_SiPM2_proj_3.pdf}
%\end{adjustwidth}
%\mbox{}\hfill\raisebox{-\height}[0pt][0pt]{\includegraphics[width=0.15\linewidth,angle=-90]{../PanelFigures/panel2_scanatminus00cm.pdf}}
%  \vspace*{-\baselineskip}
%\lenitem{Projection 3, 0.0 cm from fiber} \\
%\lenitem{LED scans with coated panel are extremely sensitive to alignment}
%\end{frame}




\begin{frame}{Establishing consistency - LED scans}
\begin{adjustwidth}{-2em}{-2em}
\includegraphics[width=0.5\linewidth]{../Figures/SCANCOMP_20150911-A1_LED_Comp_SiPM1_proj_4.pdf}
\includegraphics[width=0.5\linewidth]{../Figures/SCANCOMP_20150911-A1_LED_Comp_SiPM2_proj_4.pdf}
\end{adjustwidth}
Projection 4, 0.5 cm from fiber
\hfill \includegraphics[width=0.15\linewidth,angle=-90,right]{../PanelFigures/panel2_scanatplus05cm.pdf}
\end{frame}

\begin{frame}{Establishing consistency - LED scans}
\begin{adjustwidth}{-2em}{-2em}
\includegraphics[width=0.5\linewidth]{../Figures/SCANCOMP_20150911-A1_LED_Comp_SiPM1_proj_5.pdf}
\includegraphics[width=0.5\linewidth]{../Figures/SCANCOMP_20150911-A1_LED_Comp_SiPM2_proj_5.pdf}
\end{adjustwidth}
Projection 5, 1.0 cm from fiber
\hfill \includegraphics[width=0.15\linewidth,angle=-90,right]{../PanelFigures/panel2_scanatplus10cm.pdf}
\end{frame}

\begin{frame}{Establishing consistency - LED scans}
\begin{adjustwidth}{-2em}{-2em}
\includegraphics[width=0.5\linewidth]{../Figures/SCANCOMP_20150911-A1_LED_Comp_SiPM1_proj_6.pdf}
\includegraphics[width=0.5\linewidth]{../Figures/SCANCOMP_20150911-A1_LED_Comp_SiPM2_proj_6.pdf}
\end{adjustwidth}
Projection 6, 1.5 cm from fiber
\hfill \includegraphics[width=0.15\linewidth,angle=-90,right]{../PanelFigures/panel2_scanatplus15cm.pdf}
\end{frame}

\begin{frame}{Establishing consistency - LED scans}
\begin{adjustwidth}{-2em}{-2em}
\includegraphics[width=0.5\linewidth]{../Figures/SCANCOMP_20150911-A1_LED_Comp_SiPM1_proj_7.pdf}
\includegraphics[width=0.5\linewidth]{../Figures/SCANCOMP_20150911-A1_LED_Comp_SiPM2_proj_7.pdf}
\end{adjustwidth}
Projection 7, 2.0 cm from fiber
\hfill \includegraphics[width=0.15\linewidth,angle=-90,right]{../PanelFigures/panel2_scanatplus20cm.pdf}
\end{frame}

\begin{frame}{Establishing consistency - LED scans}
\begin{adjustwidth}{-2em}{-2em}
\includegraphics[width=0.5\linewidth]{../Figures/SCANCOMP_20150911-A1_LED_Comp_SiPM1_proj_8.pdf}
\includegraphics[width=0.5\linewidth]{../Figures/SCANCOMP_20150911-A1_LED_Comp_SiPM2_proj_8.pdf}
\end{adjustwidth}
Projection 8, 2.5 cm from fiber
\hfill \includegraphics[width=0.15\linewidth,angle=-90,right]{../PanelFigures/panel2_scanatplus25cm.pdf}
\end{frame}

%%%
%%%
%%%

\begin{frame}{2-component lightyield}
\begin{adjustwidth}{-2em}{-2em}
\includegraphics[width=0.5\linewidth]{../Figures/FITATOGETHER_20150911-1607_A1_LED_p0.pdf}
\includegraphics[width=0.5\linewidth]{../Figures/FITASYMMETRY_20150911-1607_A1_LED_p0.pdf}
\end{adjustwidth}
Projection 0, -1.5 cm from fiber
\hfill \includegraphics[width=0.15\linewidth,angle=-90,right]{../PanelFigures/panel2_scanatminus15cm.pdf}
\end{frame}

\begin{frame}{2-component lightyield}
\begin{adjustwidth}{-2em}{-2em}
\includegraphics[width=0.5\linewidth]{../Figures/FITATOGETHER_20150911-1607_A1_LED_p1.pdf}
\includegraphics[width=0.5\linewidth]{../Figures/FITASYMMETRY_20150911-1607_A1_LED_p1.pdf}
\end{adjustwidth}
Projection 1, -1.0 cm from fiber
\hfill \includegraphics[width=0.15\linewidth,angle=-90,right]{../PanelFigures/panel2_scanatminus10cm.pdf}
\end{frame}

\begin{frame}{2-component lightyield}
\begin{adjustwidth}{-2em}{-2em}
\includegraphics[width=0.5\linewidth]{../Figures/FITATOGETHER_20150911-1607_A1_LED_p2.pdf}
\includegraphics[width=0.5\linewidth]{../Figures/FITASYMMETRY_20150911-1607_A1_LED_p2.pdf}
\end{adjustwidth}
Projection 2, -0.5 cm from fiber
\hfill \includegraphics[width=0.15\linewidth,angle=-90,right]{../PanelFigures/panel2_scanatminus05cm.pdf}
\end{frame}

\begin{frame}{2-component lightyield}
\begin{adjustwidth}{-2em}{-2em}
\includegraphics[width=0.5\linewidth]{../Figures/FITATOGETHER_20150911-1607_A1_LED_p3.pdf}
\includegraphics[width=0.5\linewidth]{../Figures/FITASYMMETRY_20150911-1607_A1_LED_p3.pdf}
\end{adjustwidth}
Projection 3, 0.0 cm from fiber
\hfill \includegraphics[width=0.15\linewidth,angle=-90,right]{../PanelFigures/panel2_scanatminus00cm.pdf}
\end{frame}

\begin{frame}{2-component lightyield}
\begin{adjustwidth}{-2em}{-2em}
\includegraphics[width=0.5\linewidth]{../Figures/FITATOGETHER_20150911-1607_A1_LED_p4.pdf}
\includegraphics[width=0.5\linewidth]{../Figures/FITASYMMETRY_20150911-1607_A1_LED_p4.pdf}
\end{adjustwidth}
Projection 4, 0.5 cm from fiber
\hfill \includegraphics[width=0.15\linewidth,angle=-90,right]{../PanelFigures/panel2_scanatplus05cm.pdf}
\end{frame}

\begin{frame}{2-component lightyield}
\begin{adjustwidth}{-2em}{-2em}
\includegraphics[width=0.5\linewidth]{../Figures/FITATOGETHER_20150911-1607_A1_LED_p5.pdf}
\includegraphics[width=0.5\linewidth]{../Figures/FITASYMMETRY_20150911-1607_A1_LED_p5.pdf}
\end{adjustwidth}
Projection 5, 1.0 cm from fiber
\hfill \includegraphics[width=0.15\linewidth,angle=-90,right]{../PanelFigures/panel2_scanatplus10cm.pdf}
\end{frame}

\begin{frame}{2-component lightyield}
\begin{adjustwidth}{-2em}{-2em}
\includegraphics[width=0.5\linewidth]{../Figures/FITATOGETHER_20150911-1607_A1_LED_p6.pdf}
\includegraphics[width=0.5\linewidth]{../Figures/FITASYMMETRY_20150911-1607_A1_LED_p6.pdf}
\end{adjustwidth}
Projection 6, 1.5 cm from fiber
\hfill \includegraphics[width=0.15\linewidth,angle=-90,right]{../PanelFigures/panel2_scanatplus15cm.pdf}
\end{frame}

\begin{frame}{2-component lightyield}
\begin{adjustwidth}{-2em}{-2em}
\includegraphics[width=0.5\linewidth]{../Figures/FITATOGETHER_20150911-1607_A1_LED_p7.pdf}
\includegraphics[width=0.5\linewidth]{../Figures/FITASYMMETRY_20150911-1607_A1_LED_p7.pdf}
\end{adjustwidth}
Projection 7, 2.0 cm from fiber
\hfill \includegraphics[width=0.15\linewidth,angle=-90,right]{../PanelFigures/panel2_scanatplus20cm.pdf}
\end{frame}

\begin{frame}{2-component lightyield}
\begin{adjustwidth}{-2em}{-2em}
\includegraphics[width=0.5\linewidth]{../Figures/FITATOGETHER_20150911-1607_A1_LED_p8.pdf}
\includegraphics[width=0.5\linewidth]{../Figures/FITASYMMETRY_20150911-1607_A1_LED_p8.pdf}
\end{adjustwidth}
Projection 8, 2.5 cm from fiber
\hfill \includegraphics[width=0.15\linewidth,angle=-90,right]{../PanelFigures/panel2_scanatplus25cm.pdf}
\end{frame}



\begin{frame}{2-component lightyield}
Comparison of projections
\begin{center}
\includegraphics[width=0.75\linewidth]{../Figures/20150911-1607_A1_LED_AsymmetryComparison.pdf}
\end{center}
\end{frame}


%%%
%%%
%%%



\end{document}


\begin{frame}{Landau distribution?}
Is 2000 events enough?
\begin{center}
\includegraphics[width=0.75\linewidth]{../Figures/Distribution/20150910-1357_random.pdf} \\
\end{center}
Yes, as long as the distribution is close to a true Landau
\end{frame}



\begin{frame}{Landau distribution?}
Does the distribution from the Strontium-90 source look like a Landau?
\begin{center}
\includegraphics[width=0.75\linewidth]{../Figures/Distribution/20150910-1357_boltage.pdf} \\
\end{center}
Kinda but not really?
\end{frame}


\begin{frame}{2-component light yield modeling---Combined SiPMs Asymmetry}
When comparing the light yields from the SiPMs on opposite sides of the panels,
we can calculate the asymmetry between the yields so that the normalization
cancels \\
\begin{itemize}
\item Let $Y_1$ denote the light yield from SiPM1, and similarly for $Y_2$
\item We define the asymmetry as $A = (Y_2-Y_1)/(Y_2+Y_1)$
\end{itemize}
\vspace{10pt}
Assuming a two-component model for the light yield,
we can perform a fit to the data to extract the light fraction in the core \\
\begin{itemize}
%%%\item Let $f_{core}$ be the light fraction in the core, $L$ be the length of the panel, $\lambda_{core}$ be the attenuation
%%%length in the fiber (3.5 m for Y11 non S-type) and $\lambda_{\textnormal{clad}}$
%%%be the attenuation length in the cladding (estimated 4 cm)
%%%\item $Y_1 = f_{core}e^{-x/\lambda_{core}} + (1-f_{core})e^{-x/\lambda_{\textnormal{clad}}}$
%%%\item $Y_2 = f_{core}e^{(x-L)/\lambda_{core}} + (1-f_{core})e^{(x-L)/\lambda_{\textnormal{clad}}}$
%%%%\item $Y_2 = fe^{(x-25~\textnormal{cm})/\lambda_{core}} + (1-f)e^{(x-25~\textnormal{cm})/\lambda_{\textnormal{clad}}}$
\item Let $f_{\textnormal{core}}$ be the light fraction in the core, $L$ be the length of the panel, $\lambda_{\textnormal{core}}$ be the attenuation
length in the fiber (3.5 m for Y11 non S-type) and $\lambda_{\textnormal{clad}}$
be the attenuation length in the cladding (estimated 4 cm)
\item $Y_1 = f_{\textnormal{core}}e^{-x/\lambda_{\textnormal{core}}} + (1-f_{\textnormal{core}})e^{-x/\lambda_{\textnormal{clad}}}$
\item $Y_2 = f_{\textnormal{core}}e^{(x-L)/\lambda_{\textnormal{core}}} + (1-f_{\textnormal{core}})e^{(x-L)/\lambda_{\textnormal{clad}}}$
\end{itemize}
%\vspace{10pt}
%For this to really work, each SiPM needs to have almost exactly the same gain \\
%Our SiPMs are gain matched and are within a few tenths of a percent
\end{frame}




\begin{frame}{Establishing consistency - Source scans}
Distribution of Foreground and Background
\begin{center}
\includegraphics<1>[width=0.75\linewidth]{../Figures/Distribution/20150915_FGBG_pe.pdf}
\includegraphics<2>[width=0.75\linewidth]{../Figures/Distribution/20150915_FGBG_pe_sub.pdf}
\includegraphics<3>[width=0.75\linewidth]{../Figures/Distribution/20150915_FGBG_log_pe_sub.pdf}
\end{center}
Event count: 2,000 \\
Foreground rate: 1038 Hz \\
Background rate: 27 Hz \\
Background percentage: 2.6\%
\end{frame}

\begin{frame}{Establishing consistency - Source scans}
Distribution of Foreground and Background
\begin{center}
\includegraphics<1>[width=0.75\linewidth]{../Figures/Distribution/20150921_FGBG_pe.pdf}
\includegraphics<2>[width=0.75\linewidth]{../Figures/Distribution/20150921_FGBG_pe_sub.pdf}
\includegraphics<3>[width=0.75\linewidth]{../Figures/Distribution/20150921_FGBG_log_pe_sub.pdf}
\end{center}
Event count: 20,000 \\
Foreground rate: 1038 Hz \\
Background rate: 27 Hz \\
Background percentage: 2.6\%
\end{frame}



\begin{frame}
Executive summary of part 3---good consistency \\
Interesting to observe that the ordering is consistent throughout \\
Could it be the weather? \\
\vspace{10pt}
\begin{tabular}{|l|l|l|l|l|l|}
\hline
Color code & Date & Time & Temperature & Rel. humidity & Atm. pressure \\
\hline
Black & 9/11 & 12pm & 64~$^{\circ}$F & 46\% & 30.30 inHg \\
Red   & 9/11 &  2pm & 70~$^{\circ}$F & 43\% & 30.28 inHg \\
Blue  & 9/14 & 12pm & 84~$^{\circ}$F & 14\% & 29.91 inHg \\
\hline
\end{tabular} \\
\vspace{10pt}
All weather metrics order Black-Red-Blue while the data order Red-Black-Blue
\end{frame}
