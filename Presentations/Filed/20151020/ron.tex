\documentclass[compress,8pt]{beamer} %%%
\usetheme{Warsaw} %%%

%\setbeamersize{text margin left=10pt,text margin right=10pt}

\usepackage{changepage}

\newcommand{\lenitem}[2][.7\linewidth]{\parbox[t]{#1}{\strut #2\strut}}

%\newcommand\Fontvi{\fontsize{6}{7.2}\selectfont} %%% doesn't hel for subbullets :(
%\setlength{\columnsep}{-20pt} % doesn't work

%%% define my own headline
\setbeamertemplate{headline}
{%
  \leavevmode%
  \begin{beamercolorbox}[wd=\paperwidth,ht=2.5ex,dp=1.125ex]{section in head/foot}%
    \insertsectionnavigationhorizontal{\paperwidth}{\hskip0pt}{}%
  \end{beamercolorbox}%
  \vskip2pt%
}

%%% define my own footline
\setbeamertemplate{footline}
{%
  \leavevmode%
  \hbox{\begin{beamercolorbox}[wd=.5\paperwidth,ht=2.5ex,dp=1.125ex,leftskip=.3cm plus1fill,rightskip=.3cm]{author in head/foot}%
    \usebeamerfont{author in head/foot}\insertshortauthor
  \end{beamercolorbox}%
  \begin{beamercolorbox}[wd=.5\paperwidth,ht=2.5ex,dp=1.125ex,leftskip=.3cm,rightskip=.3cm plus1fil]{title in head/foot}%
    \usebeamerfont{title in head/foot}\insertshorttitle
  \end{beamercolorbox}}%
  \vskip0pt%
}

%%% set another another option
\setbeamertemplate{navigation symbols}{} % this is the same as \beamertemplatenavigationsymbolsempty

%%% add some packages
\usepackage{amsmath,amssymb,amsfonts,amsthm,amsbsy} % math symbols and the like
\usepackage{slashed} % Feynman slashed notatation
\usepackage{tikz}
\usepackage{pdfpages} % include pdfs as a whole page

%%% Cyrillic fonts
\input cyracc.def
%\font\tencyr=wncyr10
\font\tencyr=wncyr9
\def\cyr{\tencyr\cyracc}

%makes footnote with symbol instead of number
\long\def\symbolfootnote[#1]#2{\begingroup\def\thefootnote{\fnsymbol{footnote}}\footnote[#1]{#2}\endgroup}

%\symbolfootnote[sym #]{footnote text}     %makes footnote with symbol instead of number
%\long\def\symbolfootnote[#1]#2{\begingroup
%\def\thefootnote{\fnsymbol{footnote}}\footnotetext[#1]{#2}\endgroup}

\newcommand{\pd}[2]{\frac{\partial #1}{\partial #2}} %% Creates a partial derivative
\newcommand{\totd}[2]{\frac{d #1}{d #2}} %% Creates a total derivative


%%% title and basic information
\title[sPHENIX HCal meeting, Oct 20, 2015 - Slide \insertframenumber]{Test Stand Status}
\author[CU-Boulder]{Sebastian Vazquez-Torres \\  Ron Belmont \\ Jamie Nagle \\ \vspace{20pt} University of Colorado, Boulder}
\date{October 20\textsuperscript{th}, 2015}


%%%%%%%%%%%%%%%%%%%%%%%%%%%%%% remove line above footnotes...
\renewcommand{\footnoterule}{
  \kern -3pt
  \hrule width \textwidth height 0pt
  \kern 3pt
}


%\titlegraphic{\includegraphics[height=1.0cm]{logos/sphenixlogo.pdf}\hfill\includegraphics[trim=0 44 0 30, clip=true, height=1.0cm]{logos/Boulder_FL.pdf}}
%\titlegraphic{\includegraphics[height=1.0cm]{logos/sphenixlogo.pdf}\hfill\includegraphics[trim=0 30 0 30, clip=true, height=1.0cm]{logos/Boulder_FL.pdf}}
\titlegraphic{\includegraphics[height=1.2cm]{logos/sphenixlogo.pdf}\hfill\includegraphics[trim=0 30 0 30, clip=true, height=1.2cm]{logos/Boulder_FL.pdf}}


%%% - now begin document
\begin{document}


%%% - title slide
\begin{frame}
\titlepage
\end{frame}






\begin{frame}{Cosmics}
Some data about polystyrene (scintillator base material) \\
\begin{tabular}{ll}
\hline
$Z/A$                                                 & 0.53768 mol g$^{-1}$ \\
%$Z$                                                   & 6.0000               \\
%$A$                                                   & 11.1592 g mol$^{-1}$ \\
Density $\rho$                                        & 1.060 g cm$^{-3}$    \\
Nuclear interaction length $\lambda_I$                & 77.1 cm              \\
Radiation length $X_0$                                & 41.31 cm             \\
Mean excitation energy $I$                            & 68.7 eV              \\
Minimum ionization energy $dE/dx|_{min}$              & 2.025 MeV/cm         \\
\hline
\end{tabular}
\end{frame}



\begin{frame}{Cosmics}
What about the Bethe formula?
\begin{equation*}
-\left<\frac{dE}{dx}\right> = \rho K\frac{Z}{A}\frac{1}{\beta^2}
\left[ \frac{1}{2} \ln \frac{2m_ec^2\beta^2\gamma^2T_{max}}{I^2} - \beta^2 - \frac{\delta(\beta\gamma)}{2}\right]
\end{equation*}
\begin{itemize}
%\item $K = 4\pi N_A r_e^2 m_ec^2 = 0.307075~\textnormal{MeV}~\textnormal{g}^{-1}~\textnormal{cm}^2$
\item $K = 4\pi N_A r_e^2 m_ec^2 = 0.307075~\textnormal{MeV}~\textnormal{mol}^{-1}~\textnormal{cm}^2$
\item $T_{max} = 2m_ec^2\beta^2\gamma^2/[1 + 2\gamma m_e/M + (m_e/M)^2]$
is the highest kinetic energy that can be imparted to a free electron in a single collision
by a particle with mass $M$
\item The $\delta(\beta\gamma)$ can be calculated based on Sternheimer et al, Phys. Rev. B 26, 6067
(1982)---note that in their convention the correction doesn't have the factor 1/2 \\
$X = \log_{10} \beta\gamma$ \\
$X < X_0 \rightarrow \delta(X) = 0$ \\
$X_0 < X < X_1 \rightarrow \delta(X) = 4.6052X + a(X_1-X)^m + C$ \\
$X_1 < X \rightarrow \delta(X) = 4.6052X + C$ \\
where $X_0$, $X_1$, $a$, $m$, and $C$ are material-specific constants
\item Minimum ionizing energy for muon is 318 MeV, using this we obtain $\langle dE/dx\rangle$ = 2.036 MeV/cm,
which is quite close to the 2.025 MeV/cm from the data table
\end{itemize}
\end{frame}



\begin{frame}{Cosmics}
Generally we consider the distribution of energies to be a Landau \\
The Landau distribution is defined by
\begin{equation*}
f(\lambda) = \frac{1}{2\pi i} \int_{c+i\infty}^{c+i\infty} e^{s \ln s + \lambda s} ds
\end{equation*}
due to the long tail, the moments are undefined. \\
\vspace{10pt}
However, as derived in  J. E. Moyal, Phil. Mag. 46 (1955) 263, the Landau distribution can be well approximated by
\begin{equation*}
f(\lambda) = \frac{1}{\sqrt{2\pi}} e^{-\lambda+e^{-\lambda}}
\end{equation*}
which later became known as the Gumbel distribution. \\
The moments of the Gumbel distribution are defined:
%for $\lambda = (x-\mu)/\sigma$, the mode (MPV) is $\mu$ and the mean is $\mu + 0.577216\sigma$ \\
for $\lambda = (x-\mu)/\sigma$, the mode (MPV) is $\mu$ and the mean is $\mu + \gamma_E\sigma$
(where $\gamma_E \approx$ 0.577216 is the Euler-Mascheroni constant)
%\vspace{10pt}
%\tiny{
%NB---Gumbel was a mathematician and Anti-Nazi political writer, exiled from Germany in 1932
%}
\end{frame}



\begin{frame}{Cosmics}
For energy loss in a material, we can write the energy loss distribution as
\begin{equation*}
%\frac{dE}{dx} = \frac{1}{\sqrt{2\pi}} e^{-\lambda+e^{-\lambda}}
\Delta E = \frac{1}{\sqrt{2\pi}} e^{-\lambda+e^{-\lambda}}
\end{equation*}
where the independent variable $\lambda$ is
\begin{equation*}
\lambda = \frac{\Delta E - [\Delta E]_{MPV}}{\xi}
\end{equation*}
the width parameter $\xi$ is
\begin{equation*}
%\xi = \rho K\frac{Z}{A}\frac{1}{\beta^2}\Delta x
\xi =  \frac{K}{2}\frac{Z}{A}\frac{1}{\beta^2}\rho\Delta x
\end{equation*}
and the most probable value of the energy loss can be determined by a modified Bethe formula
\begin{equation*}
[\Delta E]_{MPV} = \xi \left[ \ln \frac{2m_ec^2\beta^2\gamma^2\xi}{I^2} -\beta^2 + 1 -\gamma_E \right]
\end{equation*}
\end{frame}



\begin{frame}{Cosmics}
Some photographs of the cosmics set up
\begin{adjustwidth}{-2em}{-2em}
\begin{center}
\includegraphics[width=0.5\linewidth]{../Photos/over.jpg}
\includegraphics[width=0.5\linewidth]{../Photos/under.jpg}
\end{center}
\end{adjustwidth}
We have both phototubes under the table so that source/LED scans can be run without total deconstruction \\
We have the upper tube as close to the panel as possible to minimize the fraction of particles that
trigger both tubes but miss the panel
\end{frame}



\begin{frame}{Cosmics}
A little about the geometry...
\begin{itemize}
\item The two phototubes are separated by about 3.25", each having 0.25" thickness, and each has a 1" $\times$ 1" scintillator
\item The maximum angle off-normal is
$\theta_{max} = \tan^{-1}(3.75/\sqrt{2}) \approx 0.361 \approx 20.7^{\circ} $
\item The pathlength $L$ is related to the thickness $\Delta x$ by $L = \Delta x \sec \theta$,
meaning $L_{max} = \Delta x \sec \theta_{max} \approx 1.07\Delta x$
\item Panel thickness 0.300" = 0.762 cm
\item If we ignore the 7\% possible deviation in pathlength,
%and assume $E_{loss,min} = dE/dx|_{min} \times T =$ 1.56 MeV
for MIP we get $[\Delta E]_{MPV} =$ 1.37 MeV and $\xi =$ 0.0750 MeV
\item However, since we know the angular distribution of cosmics is $\propto \cos^2\theta$,
we can compute a weighted average for the angle, $\theta_{ave} =$ 0.173, yielding average
pathlength $L_{ave} = 1.02 \Delta x$ = 0.777 cm
\item Using that, we get $[\Delta E]_{MPV} =$ 1.39 MeV and $\xi =$ 0.0765 MeV
\end{itemize}
\end{frame}



\begin{frame}{Cosmics}
\begin{adjustwidth}{-2em}{-2em}
\begin{center}
\includegraphics[width=0.75\linewidth]{../Cosmics/20151009-1743_templowffit.pdf}
\end{center}
\end{adjustwidth}
$[\Delta E]_{MPV} =$ 43.9$\pm$0.5 photoelectrons and $\xi =$ 9.3$\pm$0.2 photoelectrons \\
$[\Delta E]_{MPV} =$ 1.39 MeV (from the previous slide) implies a conversion of 31.6 pe/MeV
\end{frame}



\begin{frame}{Cosmics and Source}
What can we learn by comparing the distribution from the Strontium-90 source to the cosmics?
\end{frame}


\begin{frame}{Cosmics and Source}
\begin{adjustwidth}{-2em}{-2em}
\begin{center}
\includegraphics[width=0.5\linewidth]{../Cosmics/20151009-1743_templowffit.pdf}
\includegraphics[width=0.5\linewidth]{../Source/source_lskewgengaus_sum.pdf}
\end{center}
\end{adjustwidth}
%Gumbel for cosmics, modified Gaussian for source
\end{frame}



\begin{frame}{Source}
What does the Strontium source spectrum look like?
\begin{adjustwidth}{-2em}{-2em}
\begin{center}
\includegraphics[width=0.5\linewidth]{../Source/strontium_kurie.pdf}
\includegraphics[width=0.5\linewidth]{../Source/strontium_kurie2p2F.pdf}
\end{center}
\end{adjustwidth}
Data from W. E. Meyerhof, Phys. Rev. 74 (1948) 263
\end{frame}



\begin{frame}{Source}
Overlay of the source spectrum with measured distribution
\begin{adjustwidth}{-2em}{-2em}
\begin{center}
\includegraphics[width=0.5\linewidth]{../Source/monday4.pdf}
\includegraphics[width=0.5\linewidth]{../Source/ratiomonday4.pdf}
\end{center}
\end{adjustwidth}
There seems to be some inefficiency at low energies, and the ratio resembles a
turn-on curve---further investigation is needed \\
Reminder: we use the SiPMs to self-trigger on the source, so there's
an inherent low energy cut-off
\end{frame}



\begin{frame}{Brief summary}
\begin{itemize}
\item \textbf{Cosmics have been measured with a mean of 43.9 pe and a with of 9.3 pe,
with an implied conversion of 31.6 pe/MeV}
\item \textbf{We are prepared and able to cosmics measurements as soon as we get the full size tiles from BNL}
\item \textbf{We are seeking input to going further using cosmics to characterize the energy deposited in the tile}
\item We can also do an LED scan as soon as we receive them---we plan to do LED scans with the tile inverted,
i.e. facing away from the LED so that light going directly into the fiber isn't an issue
\item It may also be possible to use the Strontium-90 source to calibrate the energy, though further investigation is needed
\item The sampling of the light in the fiber relative to the total light produced may be an important additional consideration
\end{itemize}
\end{frame}



\begin{frame}
Extra material
\end{frame}


\begin{frame}{Source}
Source distribution, comparison between each SiPM and the sum
\begin{adjustwidth}{-2em}{-2em}
\begin{center}
\includegraphics[width=0.5\linewidth]{../Source/source_sbpe.pdf}
\includegraphics[width=0.5\linewidth]{../Source/source_sum.pdf}
\end{center}
\end{adjustwidth}
\end{frame}



\begin{frame}{Cosmics and Source}
\begin{adjustwidth}{-2em}{-2em}
\begin{center}
\includegraphics[width=0.5\linewidth]{../Cosmics/20151009-1743_templow.pdf}
\includegraphics[width=0.5\linewidth]{../Source/source_sum.pdf}
\end{center}
\end{adjustwidth}
Raw distributions
\end{frame}



\begin{frame}{Cosmics and Source}
\begin{adjustwidth}{-2em}{-2em}
\begin{center}
\includegraphics[width=0.5\linewidth]{../Cosmics/20151009-1743_templowfit.pdf}
\includegraphics[width=0.5\linewidth]{../Source/source_gaus_sum.pdf}
\end{center}
\end{adjustwidth}
Built-in Landau for cosmics, Built-in Gaussian for source
\end{frame}



\begin{frame}{Cosmics and Source}
\begin{adjustwidth}{-2em}{-2em}
\begin{center}
\includegraphics[width=0.5\linewidth]{../Cosmics/20151009-1743_templowffit.pdf}
\includegraphics[width=0.5\linewidth]{../Source/source_lskewgengaus_sum.pdf}
\end{center}
\end{adjustwidth}
Gumbel for cosmics, modified Gaussian for source
\end{frame}





\end{document}

