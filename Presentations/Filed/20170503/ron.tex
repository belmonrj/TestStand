\documentclass[compress,8pt]{beamer} %%%
\usetheme{Warsaw} %%%
%\usecolortheme{cu} %%%

%\setbeamersize{text margin left=10pt,text margin right=10pt}

\usepackage{changepage}

\newcommand{\lenitem}[2][.7\linewidth]{\parbox[t]{#1}{\strut #2\strut}}

%\newcommand\Fontvi{\fontsize{6}{7.2}\selectfont} %%% doesn't hel for subbullets :(
%\setlength{\columnsep}{-20pt} % doesn't work

%%% define my own headline
\setbeamertemplate{headline}
{%
  \leavevmode%
  \begin{beamercolorbox}[wd=\paperwidth,ht=2.5ex,dp=1.125ex]{section in head/foot}%
    \insertsectionnavigationhorizontal{\paperwidth}{\hskip0pt}{}%
  \end{beamercolorbox}%
  \vskip2pt%
}

%%% define my own footline
\setbeamertemplate{footline}
{%
  \leavevmode%
  \hbox{\begin{beamercolorbox}[wd=.5\paperwidth,ht=2.5ex,dp=1.125ex,leftskip=.3cm plus1fill,rightskip=.3cm]{author in head/foot}%
    \usebeamerfont{author in head/foot}\insertshortauthor
  \end{beamercolorbox}%
  \begin{beamercolorbox}[wd=.5\paperwidth,ht=2.5ex,dp=1.125ex,leftskip=.3cm,rightskip=.3cm plus1fil]{title in head/foot}%
    \usebeamerfont{title in head/foot}\insertshorttitle
  \end{beamercolorbox}}%
  \vskip0pt%
}

%%% set another another option
\setbeamertemplate{navigation symbols}{} % this is the same as \beamertemplatenavigationsymbolsempty

%%% add some packages
\usepackage{amsmath,amssymb,amsfonts,amsthm,amsbsy} % math symbols and the like
\usepackage{slashed} % Feynman slashed notatation
\usepackage{tikz}
\usepackage{pdfpages} % include pdfs as a whole page

%%% Cyrillic fonts
\input cyracc.def
%\font\tencyr=wncyr10
\font\tencyr=wncyr9
\def\cyr{\tencyr\cyracc}

%makes footnote with symbol instead of number
\long\def\symbolfootnote[#1]#2{\begingroup\def\thefootnote{\fnsymbol{footnote}}\footnote[#1]{#2}\endgroup}

%\symbolfootnote[sym #]{footnote text}     %makes footnote with symbol instead of number
%\long\def\symbolfootnote[#1]#2{\begingroup
%\def\thefootnote{\fnsymbol{footnote}}\footnotetext[#1]{#2}\endgroup}

\newcommand{\pd}[2]{\frac{\partial #1}{\partial #2}} %% Creates a partial derivative
\newcommand{\totd}[2]{\frac{d #1}{d #2}} %% Creates a total derivative
\newcommand{\bracket}[1]{\left< #1 \right>}
\newcommand{\covariance}[2]{\left< #1 #2 \right> - \left< #1 \right>\left< #2 \right>}

\newcommand{\la}{\langle}
\newcommand{\ra}{\rangle}
\newcommand{\lb}{\left<}
\newcommand{\rb}{\right>}

\newcommand{ \eps }{\varepsilon}
\newcommand{ \mean }[1]{\la #1 \ra}
\newcommand{ \dmean }[1]{\la\la #1 \ra\ra}
\newcommand{ \sump }{\sideset{}{'}\sum}

\newcommand{\vt}{v_2}
\newcommand{\vtt}{v_2\{2\}}
\newcommand{\vtf}{v_2\{4\}}
\newcommand{\vtg}{v_2\{2,|\Delta\eta|>2\}}
\newcommand{\vn}{v_n}
\newcommand{\vnt}{v_n\{2\}}
\newcommand{\vnf}{v_n\{4\}}
\newcommand{\vng}{v_n\{2,|\Delta\eta|>2\}}
\newcommand{\nfvtxt}{N_{tracks}^{FVTX}}


%%% title and basic information
%\title[d+Au BES Meeting, \today ~ - Slide \insertframenumber]{Flow measurements in the d+Au BES}
%\title[d+Au BES Meeting, \today~-~Slide \insertframenumber]{Flow measurements in the d+Au BES}
%\author[CU-Boulder]{Ron B.}
%\date{d+Au BES Meeting \\ \today}
\title[\today~-~Slide \insertframenumber]{High eta HCal tile testing}
\author[CU-Boulder]{Ron}
\date{\today}


%%%%%%%%%%%%%%%%%%%%%%%%%%%%%% remove line above footnotes...
\renewcommand{\footnoterule}{
  \kern -3pt
  \hrule width \textwidth height 0pt
  \kern 3pt
}


%\titlegraphic{\includegraphics[height=1.0cm]{logos/sphenixlogo.pdf}\hfill\includegraphics[trim=0 44 0 30, clip=true, height=1.0cm]{logos/Boulder_FL.pdf}}
%\titlegraphic{\includegraphics[height=1.0cm]{logos/sphenixlogo.pdf}\hfill\includegraphics[trim=0 30 0 30, clip=true, height=1.0cm]{logos/Boulder_FL.pdf}}
\titlegraphic{\includegraphics[height=1.2cm]{logos/sphenixlogo.pdf}\hfill\includegraphics[trim=0 30 0 30, clip=true, height=1.2cm]{logos/Boulder_FL.pdf}}
%\titlegraphic{\includegraphics[height=1.0cm]{logos/phenixlogo.pdf}\hfill\includegraphics[trim=0 30 0 30, clip=true, height=1.2cm]{logos/Boulder_FL.pdf}}


%%% - now begin document
\begin{document}


%%% - title slide


\begin{frame}
\titlepage
\end{frame}



\begin{frame}{High eta HCal tile inventory and status}
\begin{adjustwidth}{-3em}{-3em}
\begin{center}
\begin{tabular}{|lllll|}
\hline
SiPM number & Tile number & inner/outer & OK/FAIL & CU tag(s) \\
\hline
945 & 116.092.0197 & outer & FAIL & 20170413-1352, 20170414-1328 \\
956 & 113.029.0131 & outer & OK   &               \\
%976 & 115.054.0159 & outer & OK   & 20170424-1328, 20170425-1409 \\
976 & 115.054.0159 & outer & OK   &  \\
990 & 114.044.0149 & outer & OK   &               \\
\hline
939 &              & inner &      & 20170428-1400 \\
955 &              & inner &      & 20170430-1100 \\
964 & 105.017.0097 & inner & OK   & 20170430-2227 \\
965 &              & inner &      & 20170421-1522 \\
\hline
\end{tabular}
\end{center}
\end{adjustwidth}
\vspace{10pt}
A few questions:
\begin{itemize}
\item Is there some way for us to get the tile inventory numbers for the other three inner tiles?
\item Is the first number the model/type number?
\item Can we get the mechanical drawings for all these?
\end{itemize}
\end{frame}



\begin{frame}
Inner HCal tiles
\end{frame}






\begin{frame}{Inner HCal Tile Scan---939}
\begin{adjustwidth}{-3em}{-3em}
\begin{center}
%\includegraphics[width=0.85\linewidth]{./Photos/Tile_939_Scan.png}
\includegraphics[width=0.95\linewidth]{../Figures/Burn/20170428-1400_VMIN_SIPM1_RVmeanHistSub.pdf}
\end{center}
\end{adjustwidth}
\end{frame}

\begin{frame}{Inner HCal Tile Scan---955}
\begin{adjustwidth}{-3em}{-3em}
\begin{center}
%\includegraphics[width=0.85\linewidth]{./Photos/Tile_955_Scan.png}
\includegraphics[width=0.95\linewidth]{../Figures/Burn/20170430-1100_VMIN_SIPM1_RVmeanHistSub.pdf}
\end{center}
\end{adjustwidth}
\end{frame}

\begin{frame}{Inner HCal Tile Scan---964}
\begin{adjustwidth}{-3em}{-3em}
\begin{center}
%\includegraphics[width=0.85\linewidth]{./Photos/Tile_964_Scan.png}
\includegraphics[width=0.95\linewidth]{../Figures/Burn/20170430-2227_VMIN_SIPM1_RVmeanHistSub.pdf}
\end{center}
\end{adjustwidth}
\end{frame}

\begin{frame}{Inner HCal Tile Scan---965}
\begin{adjustwidth}{-3em}{-3em}
\begin{center}
%\includegraphics[width=0.85\linewidth]{./Photos/Tile_965_Scan.png}
\includegraphics[width=0.95\linewidth]{../Figures/Burn/20170421-1522_VMIN_SIPM1_RVmeanHistSub.pdf}
\end{center}
\end{adjustwidth}
\end{frame}


\begin{frame}{Comparing one of the high eta ones with one of the low eta ones}
\begin{adjustwidth}{-3em}{-3em}
\begin{center}
\includegraphics[width=0.75\linewidth]{../Figures/Burn/20170421-1522_VMIN_SIPM1_RVmeanHistSub.pdf} \\
%\includegraphics[width=0.45\linewidth]{../Figures/Burn/20151113-1313_VMIN_SIPM1_RVmeanHistSub.pdf}
\includegraphics[width=0.45\linewidth]{../Figures/Burn/20160106-1320_VMIN_SIPM1_RVmeanHistSub.pdf}
\end{center}
\end{adjustwidth}
%The response seems a lot more uniform in the high eta tiles...
\end{frame}

\begin{frame}{Procedure for estimating the relative variation}
\begin{adjustwidth}{-3em}{-3em}
\begin{center}
\includegraphics[width=0.45\linewidth]{../Figures/Burn/20170421-1522_VMIN_SIPM1_1dMeanSub.pdf}
%\includegraphics[width=0.45\linewidth]{../Figures/Burn/20170428-1400_VMIN_SIPM1_1dMeanSub.pdf} \\
%\includegraphics[width=0.45\linewidth]{../Figures/Burn/20170430-1100_VMIN_SIPM1_1dMeanSub.pdf}
%\includegraphics[width=0.45\linewidth]{../Figures/Burn/20170430-2227_VMIN_SIPM1_1dMeanSub.pdf}
%\includegraphics[width=0.45\linewidth]{../Figures/Burn/20151113-1313_VMIN_SIPM1_1dMeanSub.pdf}
\includegraphics[width=0.45\linewidth]{../Figures/Burn/20160106-1320_VMIN_SIPM1_1dMeanSub.pdf}
\end{center}
\end{adjustwidth}
\begin{itemize}
\item The off-panel background needs to be subtracted
\item No perfect procedure
\item We cut out anything below 10\% of the max value
\end{itemize}
\end{frame}

\begin{frame}{Procedure for estimating the relative variation}
\begin{adjustwidth}{-3em}{-3em}
\begin{center}
\includegraphics[width=0.45\linewidth]{../Figures/Burn/20170421-1522_VMIN_SIPM1_fit1dMeanSubTrunc.pdf}
%\includegraphics[width=0.45\linewidth]{../Figures/Burn/20170428-1400_VMIN_SIPM1_fit1dMeanSubTrunc.pdf} \\
%\includegraphics[width=0.45\linewidth]{../Figures/Burn/20170430-1100_VMIN_SIPM1_fit1dMeanSubTrunc.pdf}
%\includegraphics[width=0.45\linewidth]{../Figures/Burn/20170430-2227_VMIN_SIPM1_fit1dMeanSubTrunc.pdf}
%\includegraphics[width=0.45\linewidth]{../Figures/Burn/20151113-1313_VMIN_SIPM1_fit1dMeanSubTrunc.pdf}
\includegraphics[width=0.45\linewidth]{../Figures/Burn/20160106-1320_VMIN_SIPM1_fit1dMeanSubTrunc.pdf}
\end{center}
\end{adjustwidth}
\begin{itemize}
\item Get Mean and RMS of remainder OR do a Gaussian fit over the main region
\item High region is cladding light near SiPM
\item Low region could be background, edges, regions far from fiber, etc
\end{itemize}
%From there we can estimate in two ways: simple RMS of the whole distribution,
%or look at a Gaussian fit of where most of the light is
\end{frame}


\iffalse

\begin{frame}
Outer HCal tiles
\end{frame}

\begin{frame}{Outer HCal Tile Scan---945}
\begin{adjustwidth}{-3em}{-3em}
\begin{center}
\includegraphics[width=0.85\linewidth]{./Photos/FarEtaOuterHCalScanBrokenFiber.png}
\end{center}
\end{adjustwidth}
Known bad tile... was this intentional???
\end{frame}

\fi



\begin{frame}{Brief summary}
\begin{itemize}
\item Inner tiles all look fine and have good response uniformity (relative variation of 15--20\%)
\item Outer tile results coming soon (working through a few minor setbacks)
\item Is the tile response uniformity something that might be a KPP?  We can consider ways to further
  improve our methods for assessing the tile uniformity
\end{itemize}
\end{frame}



%

\end{document}


