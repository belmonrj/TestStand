\documentclass[compress,8pt]{beamer} %%%
\usetheme{Warsaw} %%%
\usecolortheme{cu} %%%

%\setbeamersize{text margin left=10pt,text margin right=10pt}

\usepackage{changepage}

\newcommand{\lenitem}[2][.7\linewidth]{\parbox[t]{#1}{\strut #2\strut}}

%\newcommand\Fontvi{\fontsize{6}{7.2}\selectfont} %%% doesn't hel for subbullets :(
%\setlength{\columnsep}{-20pt} % doesn't work

%%% define my own headline
\setbeamertemplate{headline}
{%
  \leavevmode%
  \begin{beamercolorbox}[wd=\paperwidth,ht=2.5ex,dp=1.125ex]{section in head/foot}%
    \insertsectionnavigationhorizontal{\paperwidth}{\hskip0pt}{}%
  \end{beamercolorbox}%
  \vskip2pt%
}

%%% define my own footline
\setbeamertemplate{footline}
{%
  \leavevmode%
  \hbox{\begin{beamercolorbox}[wd=.5\paperwidth,ht=2.5ex,dp=1.125ex,leftskip=.3cm plus1fill,rightskip=.3cm]{author in head/foot}%
    \usebeamerfont{author in head/foot}\insertshortauthor
  \end{beamercolorbox}%
  \begin{beamercolorbox}[wd=.5\paperwidth,ht=2.5ex,dp=1.125ex,leftskip=.3cm,rightskip=.3cm plus1fil]{title in head/foot}%
    \usebeamerfont{title in head/foot}\insertshorttitle
  \end{beamercolorbox}}%
  \vskip0pt%
}

%%% set another another option
\setbeamertemplate{navigation symbols}{} % this is the same as \beamertemplatenavigationsymbolsempty

%%% add some packages
\usepackage{amsmath,amssymb,amsfonts,amsthm,amsbsy} % math symbols and the like
\usepackage{slashed} % Feynman slashed notatation
\usepackage{tikz}
\usepackage{pdfpages} % include pdfs as a whole page

%%% Cyrillic fonts
\input cyracc.def
%\font\tencyr=wncyr10
\font\tencyr=wncyr9
\def\cyr{\tencyr\cyracc}

%makes footnote with symbol instead of number
\long\def\symbolfootnote[#1]#2{\begingroup\def\thefootnote{\fnsymbol{footnote}}\footnote[#1]{#2}\endgroup}

%\symbolfootnote[sym #]{footnote text}     %makes footnote with symbol instead of number
%\long\def\symbolfootnote[#1]#2{\begingroup
%\def\thefootnote{\fnsymbol{footnote}}\footnotetext[#1]{#2}\endgroup}

\newcommand{\pd}[2]{\frac{\partial #1}{\partial #2}} %% Creates a partial derivative
\newcommand{\totd}[2]{\frac{d #1}{d #2}} %% Creates a total derivative


%%% title and basic information
%\title[sPHENIX HCal meeting, Oct 20, 2015 - Slide \insertframenumber]{First results from the big tile}
\title[Jan 29, 2016 - Slide \insertframenumber]{HCal Tile Testing at Colorado}
\author[CU-Boulder]{Sebastian Vazquez-Torres \\  Ron Belmont \\ Jamie Nagle \\ \vspace{20pt} University of Colorado, Boulder}
\date{January 29\textsuperscript{th}, 2016}


%%%%%%%%%%%%%%%%%%%%%%%%%%%%%% remove line above footnotes...
\renewcommand{\footnoterule}{
  \kern -3pt
  \hrule width \textwidth height 0pt
  \kern 3pt
}


%\titlegraphic{\includegraphics[height=1.0cm]{logos/sphenixlogo.pdf}\hfill\includegraphics[trim=0 44 0 30, clip=true, height=1.0cm]{logos/Boulder_FL.pdf}}
%\titlegraphic{\includegraphics[height=1.0cm]{logos/sphenixlogo.pdf}\hfill\includegraphics[trim=0 30 0 30, clip=true, height=1.0cm]{logos/Boulder_FL.pdf}}
\titlegraphic{\includegraphics[height=1.2cm]{logos/sphenixlogo.pdf}\hfill\includegraphics[trim=0 30 0 30, clip=true, height=1.2cm]{logos/Boulder_FL.pdf}}


%%% - now begin document
\begin{document}


%%% - title slide
\begin{frame}
\titlepage
\end{frame}









\begin{frame}{Observation of Saturation}
\begin{adjustwidth}{-3em}{-3em}
\begin{center}
\includegraphics<1>[width=0.75\linewidth]{../Saturation/SaturationVersion1.pdf}
\includegraphics<2>[width=0.75\linewidth]{../Saturation/SaturationVersion2.pdf}
\end{center}
\end{adjustwidth}
Saturation occurs on high gain, so we switched to low gain \\
No saturation occurs for available LED (405 nm) range
\end{frame}



\begin{frame}{Possible issue with LED}
\begin{adjustwidth}{-3em}{-3em}
\begin{center}
\includegraphics[trim=0 30 30 0, clip=true, width=0.75\linewidth]{../Saturation/2500_LED_Kink.jpg}
\end{center}
\end{adjustwidth}
For LED (405 nm) bias voltages close to the maximum,\\ we find a distortion in the SiPM waveform \\
\end{frame}



\begin{frame}{Next steps}
\begin{itemize}
\item Swap in a different wavelength LED and see if the waveform distortion still occurs
\item Note that the shorter wavelength LEDs produce significantly lower PEs per unit LED bias voltage
\item Wire several (plan: 5) LEDs (405 nm) in parallel and re-test, see how far we can get
\end{itemize}
\end{frame}



%\begin{frame}{Brief Summary}
%\begin{itemize}
%\item Although the difference is rather small, blue (405 nm) and UV (375 nm and 361 nm) are not exactly the same
%\item The UV has a less pronounced ``valley'' in between the fiber ``peaks''
%\end{itemize}
%\end{frame}



%\begin{frame}
%Extra material
%\end{frame}




\end{document}

