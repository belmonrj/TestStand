\documentclass[12pt]{article}


\usepackage{amsmath,amssymb,amsfonts,amsthm,amsbsy} \usepackage{graphicx}


\begin{document}


\section{Test stand capabilities at the University of Colorado, Boulder}

In this section we discuss the capabilities of the test stand at the University of
Colorado, Boulder (UCB) for the purposes of testing and characterizing the HCal
scintillator tiles.


\subsection{Overview}

Figure~\ref{picture:ucbstand1} shows an overall view of the UCB test stand.  The test
stand at UCB has the capability to characterize the scintillator tiles using an LED, a
Strontium-90 radioactive source, and cosmic rays.

\begin{figure}
\begin{center}
\includegraphics[width=0.5\linewidth]{Photos/IMG_2501.jpg}
\end{center}
\caption{Overview photograph of the HCal tile test stand at the University of Colorado,
  Boulder.}
\label{picture:ucbstand1}
\end{figure}


\subsection{Tile mount and readout}

Figure~\ref{picture:ucbmount} shows a close-up view of the tile mounting system.  The
scintillator tiles are mounted on a lucite platform which is secured in place with
precision cut lucite brackets.  This mounting system ensures precise positioning and
alignment.

Also visible in Figure~\ref{picture:ucbmount} are the mounting brackets for the principle
readout device, the Silicon photomultiplier (SiPM).  The SiPMs currently employed are the
Hamamatsu S10362-11-0550C.  The scintillator tiles have Kuraray Y11 wavelength shifting
(WLS) fibers and tapped holes for mounting brackets at each end of each fiber.  The
mounting brackets are 3D printed at the UCB Integrated Teaching and Learning Laboratory.
The SiPMs are placed inside the brackets which are then screwed into the mounting holes.
The mounting brackets are marked so that the end of the fiber can be centered on the SiPM.


\begin{figure}
\begin{center}
\includegraphics[width=0.5\linewidth]{Photos/IMG_2615.jpg}
\end{center}
\caption{Close up view of the tile mounting system.}
\label{picture:ucbmount}
\end{figure}


\subsection{LED/source housing and stepper motors}

The Stronitum-90 source and LED are held in a single machined two-piece brass housing with
a cylindrical shape.  The source is mounted directly in the center of the housing in the
$r-\phi$ plane and at the top of the lower piece in the $z$-direction.  Below the source
is a precision-drilled 1 mm aperture that ensures the electrons emitted reach the
scintillator tile in a precise location and normal to the surface.  Offset to the side is
a separate mount for the LED.  This enables LED and source tests to be performed in the
same setup without making any changes and, therefore, minimizing exposure to radiation.
Figure~\ref{picture:ucbhouse} shows a close-up view of the LED/source housing.

\begin{figure}
\begin{center}
\includegraphics[width=0.5\linewidth,angle=-90]{Photos/IMG_2508.jpg}
\end{center}
\caption{Close-up view of the LED/source housing.}
\label{picture:ucbhouse}
\end{figure}

The brass housing is mounted to a two-dimensional stepper motor rig, which allows
extremely precise positioning for detailed two-dimensional imaging of the panel as
required.
%The stepper motors are Zaber NM series with the X-MBC1 controller.



\subsection{LED}

The LED is a Bivar UV3TZ-405-30, with an InGaN/Sapphire chip, peak wavelength of 405 nm,
and 30$^{\circ}$ angular spread of the emitted light.  The LED is powered by a function
generator with a 20 ns pulse-width and a 1 ${\rm \mu}$s period.  The function generator
provides an additional channel for a trigger and the data acquisition is triggered on this
channel.  The trigger rate of 1 MHz is three orders of magnitude larger than the rate of
emission of decay products from the radioactive source, so that signal contamintion is not
of concern.


\subsection{Source}

The radioactive source is Strontium-90, which has a dominant decay chain of
%\begin{equation*}
%^{90}_{38}{\rm Sr} \rightarrow {}^{90}_{39}{\rm Y} + \beta^-(0.547~{\rm MeV})
%\rightarrow {}^{90}_{40}{\rm Zr} + \beta^-(2.28~{\rm MeV}) + \beta^-(0.547~{\rm MeV}).
%\end{equation*}
$^{90}_{38}{\rm Sr} \rightarrow {}^{90}_{39}{\rm Y} + \beta^-(0.547~{\rm MeV}) \rightarrow
{}^{90}_{40}{\rm Zr} + \beta^-(2.28~{\rm MeV}) + \beta^-(0.547~{\rm MeV})$.  The
Strontium-90 half-life is 28.8 yr while the Yttrium-90 half-life is 64 hr.

The trigger used is a coincidence of both SiPMs.  The observed coincidence rate with the
source over the panel is 1.04 kHz, while the observed coincidence rate with the source far
away from the panel is 27.0 Hz, for a background fraction of 2.60\%.  The coincidence rate
with the source far away from the panel is dominated by quantum noise in the SiPMs.
%though cosmic rays and shower particles from the LED/source housing may also contribute.


\subsection{Cosmic rays}

%The city of Boulder, Colorado has an elevation of 5,360 feet above sea level and,
%therefore, the total cosmic ray flux is appreciably larger than that at sea level, by a
%factor of approximately 1.5.
Cosmic rays are triggered using two Philips XM2729 photomultiplier tubes (PMTs) with 1 cm
wide finger-style scinillators.  The PMTs are placed so that the scinillators cross at a
right angle to minimize the solid angle of accepted cosmic rays to ensure that the
selected cosmic rays are nearly perpendicular to the scintillator tile.



\end{document}
